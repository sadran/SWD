
\documentclass[english,version-2020-11]{uzl-thesis}

% Copy this file as a template for your thesis. You will have to take
% action at all places marked by
%
% !!!!!!!!!!!!!!!!!!!!!!!!!!!!!!!!!!
% !!! Your action is needed here !!!
% !!!!!!!!!!!!!!!!!!!!!!!!!!!!!!!!!!
%
% The first place your action is needed is the first line of this
% document:
%
%
% Language of the thesis:
%
% You must use either 'german' or 'english' above, depending on the
% language used in the main text. This will automatically setup a lot
% of things in the background.
%
%
% Version of the class:
%
% You must specify which version of the thesis class is to be
% used. This is important in case the class style changes in later
% years, but we still want an older thesis to look the same, even when
% things are changed in the class.
%
% Do not change or remove the version-xxxx key.
%
%
% Text encoding:
%
% Your thesis *must* be encoded in utf8 (unicode), which is the
% default in most editors these days. Do *not* change this to latin8.



%%%
%
% Main setup:
%
%%%
%
% You must use the \UzLThesisSetup command to specify numerous things
% about your thesis. This includes the entries on the title page, the 
% abstracts, and the bibliography style. You do so by specifying
% so-called "values" for so-called "keys". For instance, 
% for the key "Autor" you must provide your name as the value. You do
% so by writing 'Autor = {Max Mustermann}', that is, the value is put
% into curly braces. You can use the \UzLThesisSetup command
% repeatedly and the order in which you provide the keys is not
% important. 
%
% Everything shown on the title page must be in German -- even
% if the thesis is written in English! Just insert German text for
% German keys and English text for English keys (like 'Abstract' needs
% English text, while 'Zusammenfassung' needs German text).

\UzLThesisSetup{
  %
  % !!!!!!!!!!!!!!!!!!!!!!!!!!!!!!!!!!
  % !!! Your action is needed here !!!
  % !!!!!!!!!!!!!!!!!!!!!!!!!!!!!!!!!!
  %
  % First, specify the institut or clinic at which the thesis was
  % written. You get the logo file from them (make sure it has the
  % correct size, namely the same as the example). If they do not have
  % a logo, the university's default logo is used.
  %
  % The 'verfasst' gets two arguments. Change the first to {an der}
  % for clinics, as in 'Verfasst = {an der}{Medizinischen Klinik I}'
  %
  Logo-Dateiname        = {logos/uzl-thesis-logo-uzl.pdf},
  Verfasst              = {am}{Institut für Neuro- und Bioinformatik},
  %
  % The titles:
  %
  Titel auf Deutsch     = {
    Schätzung der Position des Lenkrads anhand von ToF-Daten
  }, 
  Titel auf Englisch    = {
    Estimation of the Steering Wheel Pose Using ToF Data
  },
  %
  % Author and supervisor:
  % 
  % Note that the 'Betreuer' or 'Betreuerin' is the supervisor, that
  % is, the professor who officially supervises the thesis. If there
  % is also an assistent of the professor who helped (typically a
  % lot), use 'Mit Unterstützung von' to thank that person. If the
  % thesis was mainly written 'externally' at some company or another
  % institute, point this out using 'Weitere Unterstützung'. 
  % 
  % For your own name, do *not* add things like "BSc" or "BSc
  % cand.". For the supervisor, you should normally include
  % "Prof. Dr." or "PD Dr." (ask your supervisor, what is
  % appropriate), but nothing more (so no
  % "Univ.-Prof. Dr. Dr. h.c. mult." unless your supervisor insists).  
  %
  Autor                 = {Sadra Nasiri},
  Betreuerin            = {Prof. Dr.-Ing. Erhardt Barth},
  % 
  % Optional: Supporting persons and institutions. The text should be
  % in German, even for an English thesis.
  %
  Mit Unterstützung von = {Die Arbeit ist im Rahmen einer Tätigkeit bei der Firma Gestigon GmbH entstanden.},
  % 
  %   Weitere Unterstützung = {
  %     Die Arbeit ist im Rahmen einer Tätigkeit bei der Firma Muster GmbH
  %     entstanden.
  %   },
  %
  %
  % Your Degree Programm (Studiengang)
  %
  % Specify 'Bachelorarbeit' or 'Masterarbeit' and the degree
  % programme. Make sure the name of programme is correct and not
  % some abbreviation or some incorrect variant. For instance:
  % 'Medizinische Ingenierwissenschaft', but not 'MIW';
  % 'Medizinische Informatik', but not 'Medizin-Informatik';
  % 'Informatik', but not 'Informatik (SSE)'.
  %
  % Use German names for German programmes and English names for
  % English ones, so 'Infection Biology', not 'Infektionsbiologie'. 
  % For programmes that have a German bachelor and an English master,
  % use the German name for a bachelor thesis and the English name for
  % the master thesis.
  %
  Masterarbeit,
  Studiengang           = {Robotics and Autonomous Systems},
  %
  % Date on which the thesis is turned in German, formatted the
  % traditional German way:
  %
  Datum                 = {15. November 2024},
  %
  % The English abstract. You must always provide abstracts in German
  % and in English. 
  %
  Abstract              = {
    This thesis develops a robust approach for estimating the 3D position and orientation of a steering wheel within a vehicle cabin using 3D point cloud data, addressing the unique challenges of confined interior spaces and frequent occlusions. The research begins by creating a specialized dataset, capturing high-quality 3D point clouds of car interiors with annotated steering wheel bounding boxes. Initial ground truth generation involved a 3D circle-fitting method, which proved ineffective due to data sparsity and noise. A refined approach using ArUco board successfully generated accurate 3D bounding boxes, providing reliable ground truth data for model training, and a comprehensive evaluation of the annotations was performed to prove their accuracy. To detect the steering wheel, this thesis leverages a modified Voxel R-CNN model, adapting the network to work with cubic point clouds and focusing on x-axis rotation, which aligns with the steering wheel’s primary movement. Experimental evaluations demonstrate the model’s effectiveness in estimating the steering wheel’s 3D position with high accuracy, achieving strong mAP scores across 3D and bird’s-eye-view (BEV) metrics. This research significantly advances in-cabin monitoring by providing a dataset and model tailored to automotive interiors, which are essential for enhancing driver monitoring and autonomous driving safety systems. 
  },
  Zusammenfassung       = {
    In dieser Arbeit wird ein robuster Ansatz zur Schätzung der 3D-Position und -Orientierung eines Lenkrads in einem Fahrzeuginnenraum anhand von 3D-Punktwolkendaten entwickelt, der die besonderen Herausforderungen von engen Innenräumen und häufigen Verdeckungen berücksichtigt. Die Forschung beginnt mit der Erstellung eines speziellen Datensatzes, der hochwertige 3D-Punktwolken von Fahrzeuginnenräumen mit beschrifteten Lenkradbegrenzungsfeldern erfasst. Die anfängliche Generierung der Grundwahrheit beinhaltete eine 3D-Kreisanpassungsmethode, die sich aufgrund der geringen Datenmenge und des Rauschens als unwirksam erwies. Ein verfeinerter Ansatz unter Verwendung des ArUco-Boards erzeugte erfolgreich genaue 3D-Begrenzungsboxen, die zuverlässige Daten für das Modelltraining lieferten. Zur Erkennung des Lenkrads wird in dieser Arbeit ein modifiziertes Voxel-R-CNN-Modell verwendet, das an die Arbeit mit kubischen Punktwolken angepasst ist und sich auf die Drehung um die x-Achse konzentriert, die mit der Hauptbewegung des Lenkrads übereinstimmt. Experimentelle Auswertungen zeigen die Effektivität des Modells bei der Schätzung der 3D-Position des Lenkrads mit hoher Genauigkeit, wobei hohe mAP-Werte für 3D- und Vogelperspektiven-Metriken (BEV) erzielt werden. Diese Forschung bringt die Überwachung im Fahrzeuginnenraum erheblich voran, indem sie einen Datensatz und ein Modell bereitstellt, die auf den Fahrzeuginnenraum zugeschnitten sind und die für die Verbesserung der Fahrerüberwachung und der Sicherheitssysteme für autonomes Fahren unerlässlich sind.
  },
  %
  % Optional: 'Danksagungen' (German) or 'Acknowledgements'
  % (English). Both keys are optional and both have the same effect of
  % adding an acknowledgements text after the abstracts and before the
  % table of contents.
  %
  Acknowledgements      = {
This thesis was conducted at Gestigon GmbH – A Valeo brand – under the supervision of Foti Coleca and Thomas Klähn. I am grateful to the team at Gestigon for their guidance and support throughout the research and development process.
I would also like to express my sincere gratitude to Professor Erhardt Barth from the Neuro-Bioinformatics Institute at the University of Lübeck for his invaluable academic supervision and insightful feedback, which greatly contributed to the progress of this work.
Additionally, I acknowledge the use of ChatGPT, an AI language model by OpenAI, for assistance with rephrasing, spell-checking, and grammar-checking throughout this document.
  },
  % Bibliography style: Choose between
  % 
  % 'Alphabetische Bibliographie'
  % for all degree programmes in the natural sciences 
  % 
  % 'Numerische Bibliographie'
  % alternative for all other degree programmes
  % 
  % Either will load biblatex and setup the citation methods and the
  % bibliography styles correctly. You should not mess with them.
  % 
  %Alphabetische Bibliographie,
  % Alternatively:
  Numerische Bibliographie
}

%%%%%%%%%%%%%%%%%%%%
%
% Styling the thesis
%
%%%%%%%%%%%%%%%%%%%%
%
% Creating a visually pleasing layout and choosing fonts is not
% easy. Furthermore, different people have different preferences. Of
% course, for the University of Lübeck, the dean of studies could just
% force everyone to use one specific layout and font, but that seems a
% bit drastic and, also, it seems nice that thesis by different people
% have an individual style even though they all stick to the same
% overall structure.
%
% For these reasons, I (Till Tantau) have spend quite some time on
% designing a flexible layout and styling mechanism for theses.
%
% Basically, the overall structure of the thesis is fixed by the
% thesis class and so are many structural elements. For instance, you
% cannot change the order in which the abstract and table of contents
% are shown, you cannot move the bibliography elsewhere, indeed, the
% bibliography style is also fixed. Likewise, the text on the title
% page is fixed.
%
% Although many things are fixed, you *can* change several other
% things. For instance, you can change the font used for the main
% text, you can change which font is used for titles and headings or
% you can change whether titles and headlines are centered or flushed
% left.
%
% There are many LaTeX packages for changing such things. You are
% kindly asked *not to use them*. Rather, use (only) the options
% offered by the thesis class. All possible choices and combinations
% there have been tested by me and produce nice results; what happens
% with other packages no one knows and might no longer conform to what
% is expected by the university. As you will see, you still have a
% lot of options.
%
%
% Technical note: All styling is done via the command
%
% \UzLStyle{...}
%
% where ... is a key-value list just as for \UzLThesisSetup. The
% difference is just that everything having to do with styling as
% controlled by \UzLStyle, while the more “formal” setup keys are
% controlled by \UzLThesisSetup.
%
%%%
%
% Designs
%
%
% A \emph{design} is a whole set of font and layout options bundled
% together. They have been chosen in such a way that a visually
% pleasing “overall appearance” results.
%
%
% \UzLStyle{computer modern oldschool design}
%
% The look of this design mimics the “classical” way a paper or report
% created with \LaTeX\ looks like: The Computer Modern font is used,
% bold face fonts are used for headlines, only black and white are
% used as colors. This design reminds me of older scientific
% documents, especially from the computer science community where
% \LaTeX\ was used very early.
%
%
% \UzLStyle{computer modern basic design}
%
% A slightly less “oldschool” version of the previous design. It is
% still a classic design in the sense that it uses the Computer Modern
% font and that it still has this “good old \LaTeX” look, but some
% more modern aspects (like colors!) have been added.
%
% Note that this design uses Myriad for the title page (one of the
% “modern aspect”), which means that his font must be installed.
%
%
% \UzLStyle{computer modern scholary design}
%
% In my opinion, this is the ultimate “scholary design”: The thesis
% will look like it has been typeset by hand some 150 years ago and
% then printed by a university press. There is really nothing “modern”
% about it and the word in the name of the design is just part of the
% name of the “Computer Modern” font.
%
%
% \UzLStyle{pagella basic design}
%
% A, well, basic design that uses the Pagella font rather than the
% Computer Modern font. Especially the bold face version of this font
% looks nicer than the Computer Modern counterpart. Also, Pagella,
% while still having a “bookish” look, still feels a bit fresher than
% Computer Modern. 
%
%
% \UzLStyle{pagella centered design}
%
% A variant of the basic Pagella design that centers all
% headlines. A nice alternative to the basic version.
%
%
% \UzLStyle{pagella contrast design}
%
% This design tries to create some visual friction by contrasting the
% sans serif headline font (in bold!) with the main text. I find it a
% visually very interesting combination.
%
%
% \UzLStyle{alegrya basic design}
%
% The third variant of the basic design, this time using the Alegrya
% font. 
%
%
% \UzLStyle{alegrya scholary design}
%
% The Alegrya version of the previous “scholary” design. Unlike the
% Computer Modern version, this design does not look old, but more
% fresh -- while still creating the impression that the text must be
% about a very scientific subject. 
%
%
% \UzLStyle{alegrya stylish design}
%
% The design is quite similar to the scholary version for the Alegrya
% font, but with even more modern additions. “Stylish” is the word
% that comes to my mind.
%
%
\UzLStyle{alegrya modern design}
%
% A design that uses the sans serif version of the Alegrya font for
% the headlines. This is a nice modern overall design.
%
%%%




%%%%%%%%
%
% Now, include the package you need here using \usepackage. 
%
% However, many standard packages are already loaded by the class:
%
% amsmath, amssymb, amsthm, babel, biblatex, csquotes, etoolbox,
% filecontents, fontspec, geometry, hyperref, tikz (with libraries
% arrows.meta, positioning and shapes), varioref, url 
%
% Indeed, in many cases you will not need any extra packages.
%
%%%%%%%
\usepackage{cleveref}
\usepackage{multirow}
\usepackage{subcaption}
\usepackage{caption}

\begin{document}

%
% The title page and table of contents will be inserted automatically
% here. 
%

\chapter{Introduction}
% In a German thesis write: \chapter{Einleitung}


The increasing complexity of modern automotive systems has led to a growing
demand for precise in-cabin monitoring technologies. Among these, 
steering wheel detection plays a pivotal role in monitoring driver
behavior, enhancing safety, and facilitating autonomous control in 
advanced driver-assistance systems. Estimating the 3D location and 
orientation of the steering wheel is crucial for systems that rely on
driver interaction and vehicle control. 
Yet, achieving accurate 3D detection of such an object within a car's 
interior presents unique challenges, including variations in rotation and
position within confined, occlusion-prone environments. This is because 
the car's interior is a complex and constrained environment, with limited 
sensor coverage and frequent occlusions of key components like the 
steering wheel. Accurately determining the 3D position and orientation of 
the steering wheel is essential for enabling advanced driver monitoring 
and assistive technologies, but poses significant technical hurdles 
compared to object detection in open, outdoor environments. 


\section{Contributions of This Thesis}
This thesis addresses the problem of steering wheel detection using 3D 
point cloud, which offers a rich representation of spatial information. 
Specifically, this research contributes in two significant ways: 

First, it introduces a novel dataset that represents the car interior, 
including 3D point clouds and annotated bounding boxes for the steering 
wheel. This dataset was created through a meticulous process involving 
the recording of point cloud data, feature extraction, and preprocessing 
to ensure data integrity and clarity. By accurately determining the 
steering wheel’s 3D position and orientation, this dataset serves as a 
critical foundation for model training and validation.

Second, this thesis proposes a model-based approach for detecting the 
steering wheel’s 3D position using a modified VoxelRCNN architecture. 
Existing 3D object detection models, such as those trained on KITTI 
dataset data, are typically optimized for outdoor environments with 
non-cubic point cloud dimensions and restricted rotational degrees of 
freedom (DOF). To overcome these limitations, the research modifies the VoxelRCNN 
architecture to account for a cubic point cloud structure and allows for rotation detection along 
the x-axis, which is crucial for accurately representing the orientation of the steering wheel.

\section{Structure of This Thesis}
This thesis is organized to guide the reader through the development of a dataset and model for 3D steering wheel detection within the context of in-cabin monitoring for autonomous driving. Chapter 1, \textbf{Introduction}, presents the motivation for this research and highlights the specific contributions of this thesis. Chapter 2, \textbf{Related Work}, provides an overview of the existing literature on 3D object detection in autonomous driving, covering detection methods, commonly used datasets, and performance metrics, along with a summary of current challenges in the field.

Chapter 3, \textbf{Creating Dataset}, details the process of building a custom dataset for steering wheel detection. This includes data collection, preprocessing, ground truth generation, and the experiments conducted to validate the dataset. Both an initial and refined approach to ground truth generation are discussed, with results presented to evaluate the dataset’s effectiveness. Chapter 4, \textbf{Estimating Steering Wheel Position}, focuses on the architecture and adaptations made to a 3D object detection network for accurately detecting the steering wheel. It includes an evaluation of the model’s performance based on the developed dataset. The thesis concludes with Chapter 5, \textbf{Conclusion}, summarizing the key findings and suggesting directions for future research.




\include{chapters/related_work}
\include{chapters/creating_dataset}
\include{chapters/estimating_steering_wheel_position}
\chapter{Conclusion}

This thesis presented a novel approach for detecting and localizing the steering wheel within 3D point clouds, specifically addressing the challenges of in-cabin automotive monitoring. The research contributed two primary innovations: the creation of the SWD dataset, capturing detailed 3D spatial representations of steering wheel positions, and the adaptation of the Voxel R-CNN model to meet the unique requirements of steering wheel detection. Together, these efforts have enabled precise and reliable estimation of both the 3D position and orientation of the steering wheel, a critical step for enhancing in-cabin monitoring and driver-assist systems.

The creation of the SWD dataset, with its tailored recording setup and structured marker arrangement, established a robust foundation for training and evaluation. The dataset’s comprehensive coverage across various angles, positions, and driver interactions provided a challenging yet realistic environment for testing detection capabilities in confined spaces. Through custom data preprocessing, such as distortion correction and noise reduction, the dataset facilitated precise ground truth generation, ensuring accuracy in evaluating the model’s spatial understanding.

Adapting Voxel R-CNN to this task involved several key modifications, including adjusting bounding box encoding to prioritize x-axis rotation and customizing the model’s 3D backbone to handle cubic point clouds effectively. These modifications addressed the specific spatial and rotational characteristics of the steering wheel, resulting in high recall and precision in detecting the steering wheel across 3D and Bird’s Eye View (BEV) projections. Evaluation metrics further validated the model’s performance, with high mAP scores, low distance error, and minimal rotation error confirming the model’s robustness in both positioning and orientation estimation.

The results of this research demonstrate the feasibility of deploying 3D object detection for in-cabin applications, particularly in scenarios where accurate steering wheel tracking is essential. The model’s ability to generalize across a variety of perspectives and orientations underscores its potential for practical use in automotive settings, contributing to safety and monitoring systems. The findings of this thesis not only highlight the adaptability of Voxel R-CNN but also lay a foundation for further advancements in 3D in-cabin monitoring.

In conclusion, this thesis provides a comprehensive approach to steering wheel detection using 3D point clouds, paving the way for enhanced in-cabin monitoring solutions. The advancements achieved here form a valuable basis for future research and application in autonomous driving and driver-assistive technology, where precise spatial awareness of in-cabin elements is increasingly vital.ß
\include{chapters/next_steps}
% Normally, the bibliography comes next at this point. Do *not* (try
% to) include further indices and tables like an index or
% a list of figures or a list of tables or such things. Nobody
% actually uses them and they just use up space. 
%
% You *can* however include a glossary, if this seems appropriate. It
% goes here as an unnumbered chapter. Most thesis will *not* need a
% glossary: a well-written text (re)explains strange words and
% concepts as necessary. However, there are situations where a
% glossary may be helpful.

\begin{bibtex-entries}

@misc{stereo_rcnn,
author = {Peiliang Li AND Xiaozhi Chen AND Shaojie Shen},
title = {Stereo R-CNN Based 3D Object Detection for Autonomous Driving},
year = {2019},
month = {06},
doi = {10.1109/cvpr.2019.00783},
url = {https://doi.org/10.1109/cvpr.2019.00783}
}
@misc{pseudo_lidar,
author = {Yan Wang AND Wei‐Lun Chao AND Divyansh Garg AND Bharath Hariharan AND Mark Campbell AND Kilian Q. Weinberger},
title = {Pseudo-LiDAR From Visual Depth Estimation: Bridging the Gap in 3D Object Detection for Autonomous Driving},
year = {2019},
month = {06},
doi = {10.1109/cvpr.2019.00864},
url = {https://doi.org/10.1109/cvpr.2019.00864}
}
@misc{pseudo_lidar++,
author = {Yurong You AND Yan Wang AND Wei‐Lun Chao AND Divyansh Garg AND Geoff Pleiss AND Bharath Hariharan AND Mark Campbell AND Kilian Q. Weinberger},
title = {Pseudo-LiDAR++: Accurate Depth for 3D Object Detection in Autonomous Driving},
journal = {Cornell University},
year = {2019},
month = {01},
doi = {10.48550/arxiv.1906.06310},
url = {https://arxiv.org/abs/1906.06310}
}
@article{dsgn,
author = {Yilun Chen AND Shu Liu AND Xiaoyong Shen AND Jiaya Jia},
title = {DSGN: Deep Stereo Geometry Network for 3D Object Detection},
year = {2020},
month = {06},
doi = {10.1109/cvpr42600.2020.01255},
url = {https://doi.org/10.1109/cvpr42600.2020.01255}
}
@article{voxelnet,
  author       = {Yin Zhou and
                  Oncel Tuzel},
  title        = {VoxelNet: End-to-End Learning for Point Cloud Based 3D Object Detection},
  journal      = {CoRR},
  volume       = {abs/1711.06396},
  year         = {2017},
  url          = {http://arxiv.org/abs/1711.06396},
  eprinttype    = {arXiv},
  eprint       = {1711.06396},
  timestamp    = {Mon, 13 Aug 2018 16:46:15 +0200},
  biburl       = {https://dblp.org/rec/journals/corr/abs-1711-06396.bib},
  bibsource    = {dblp computer science bibliography, https://dblp.org}
}
@article{voxelrcnn,
author = {Jiajun Deng AND Shaoshuai Shi AND Peiwei Li AND Wengang Zhou AND Yanyong Zhang AND Houqiang Li},
title = {Voxel R-CNN: Towards High Performance Voxel-based 3D Object Detection},
journal = {Association for the Advancement of Artificial Intelligence},
volume = {35},
number = {2},
pages = {1201-1209},
year = {2021},
month = {05},
doi = {10.1609/aaai.v35i2.16207},
url = {https://doi.org/10.1609/aaai.v35i2.16207}
}
@article{se_ssd,
author = {Zheng Wu AND Weiliang Tang AND Li Jiang AND Chi‐Wing Fu},
title = {SE-SSD: Self-Ensembling Single-Stage Object Detector From Point Cloud},
year = {2021},
month = {06},
doi = {10.1109/cvpr46437.2021.01426},
url = {https://doi.org/10.1109/cvpr46437.2021.01426}
}
@article{bdc_det,
author = {Qiangeng Xu AND Yiqi Zhong AND Ulrich Neumann},
title = {Behind the Curtain: Learning Occluded Shapes for 3D Object Detection},
journal = {Association for the Advancement of Artificial Intelligence},
volume = {36},
number = {3},
pages = {2893-2901},
year = {2022},
month = {06},
doi = {10.1609/aaai.v36i3.20194},
url = {https://doi.org/10.1609/aaai.v36i3.20194}
}
@misc{point_rcnn,
author = {Shaoshuai Shi AND Wei Wang AND Hongsheng Li},
title = {PointRCNN: 3D Object Proposal Generation and Detection from Point Cloud},
journal = {Cornell University},
year = {2018},
month = {01},
doi = {10.48550/arxiv.1812.04244},
url = {https://arxiv.org/abs/1812.04244}
}
@misc{point_gnn,
author = {Weijing Shi AND  Ragunathan AND  Rajkumar},
title = {Point-GNN: Graph Neural Network for 3D Object Detection in a Point Cloud},
journal = {Cornell University},
year = {2020},
month = {01},
doi = {10.48550/arxiv.2003.01251},
url = {https://arxiv.org/abs/2003.01251}
}
@misc{pointnet++,
author = {Charles R. Qi AND Yi Li AND Hao Su AND Leonidas Guibas},
title = {PointNet++: Deep Hierarchical Feature Learning on Point Sets in a Metric Space},
journal = {Cornell University},
year = {2017},
month = {01},
doi = {10.48550/arxiv.1706.02413},
url = {https://arxiv.org/abs/1706.02413}
}
@article{pv_rcnn,
author = {Shaoshuai Shi AND Chaoxu Guo AND Li Jiang AND Zhe Wang AND Jianping Shi AND Wei Wang AND Hongsheng Li},
title = {PV-RCNN: Point-Voxel Feature Set Abstraction for 3D Object Detection},
year = {2020},
month = {06},
doi = {10.1109/cvpr42600.2020.01054},
url = {https://doi.org/10.1109/cvpr42600.2020.01054}
}
@article{pointpainting,
author = {Sourabh Vora AND Alex Lang AND Bassam Helou AND Oscar Beijbom},
title = {PointPainting: Sequential Fusion for 3D Object Detection},
year = {2020},
month = {06},
doi = {10.1109/cvpr42600.2020.00466},
url = {https://doi.org/10.1109/cvpr42600.2020.00466}
}

@article{frustum_pointnet,
  author       = {Charles Ruizhongtai Qi and
                  Wei Liu and
                  Chenxia Wu and
                  Hao Su and
                  Leonidas J. Guibas},
  title        = {Frustum PointNets for 3D Object Detection from {RGB-D} Data},
  journal      = {CoRR},
  volume       = {abs/1711.08488},
  year         = {2017},
  url          = {http://arxiv.org/abs/1711.08488},
  eprinttype    = {arXiv},
  eprint       = {1711.08488},
  timestamp    = {Wed, 11 Nov 2020 08:48:10 +0100},
  biburl       = {https://dblp.org/rec/journals/corr/abs-1711-08488.bib},
  bibsource    = {dblp computer science bibliography, https://dblp.org}
}

@misc{avod,
author = {Jason S. Ku AND Melissa Mozifian AND Jungwook Lee AND Ali Harakeh AND Steven L. Waslander},
title = {Joint 3D Proposal Generation and Object Detection from View Aggregation},
year = {2018},
month = {10},
doi = {10.1109/iros.2018.8594049},
url = {https://doi.org/10.1109/iros.2018.8594049}
}

@misc{mv3d,
author = {Chen; Xiaozhi; Ma; Huimin; Wan; Ji; Li; Bo; Xia; Tian undefined},
title = {Multi-View 3D Object Detection Network for Autonomous Driving},
year = {2016},
month = {11},
url = {https://arxiv.org/abs/1611.07759v2}
}

@article{survey,
author = {Eduardo Arnold AND Omar Y. Al-Jarrah AND Mehrdad Dianati AND Saber Fallah AND David Oxtoby AND Alex Mouzakitis},
title = {A Survey on 3D Object Detection Methods for Autonomous Driving Applications},
journal = {Institute of Electrical and Electronics Engineers},
volume = {20},
number = {10},
pages = {3782-3795},
year = {2019},
month = {01}
}

@article{kitti,
  author = {Andreas Geiger and Philip Lenz and Christoph Stiller and Raquel Urtasun},
  title = {Vision meets Robotics: The KITTI Dataset},
  journal = {International Journal of Robotics Research (IJRR)},
  year = {2013}
}

@misc{nuscenes,
author = {Holger Caesar AND Varun Bankiti AND Alex Lang AND Sourabh Vora AND Venice Erin Liong AND Qiang Xu AND Krishnan Anush AND Pan Yu AND Giancarlo Baldan AND Oscar Beijbom},
title = {nuScenes: A multimodal dataset for autonomous driving},
journal = {Cornell University},
year = {2019},
month = {01},
doi = {10.48550/arxiv.1903.11027},
url = {https://arxiv.org/abs/1903.11027}
}

@misc{waymo,
author = {Pei Sun AND Henrik Kretzschmar AND Xerxes Dotiwalla AND Aurélien Chouard AND Vijaysai Patnaik AND Paul Tsui AND James C. Y. Guo AND Yin Zhou AND Yuning Chai AND Benjamin Caine AND Vijay Vasudevan AND Wei Han AND Jiquan Ngiam AND Hang Zhao AND Aleksei Timofeev AND Scott Ettinger AND Maxim Krivokon AND Amy Gao AND Aditya Joshi AND Yu Zhang AND Jonathon Shlens AND Zhifeng Chen AND Dragomir Anguelov},
title = {Scalability in Perception for Autonomous Driving: Waymo Open Dataset},
year = {2020},
month = {06},
doi = {10.1109/cvpr42600.2020.00252},
url = {https://doi.org/10.1109/cvpr42600.2020.00252}
}

@misc{opencv_aruco_detection,
  author       = {OpenCV},
  title        = {ArUco Marker Detection},
  year         = {2024},
  howpublished = {\url{https://docs.opencv.org/4.x/d5/dae/tutorial_aruco_detection.html}},
  note         = {Accessed: 2024-11-04}
}
\end{bibtex-entries}


% If you need to have an appendix (I advise against it), insert it
% here using, first, \appendix and then \chapter and then,
% possibly, \section. 
%
% \appendix
%
% \chapter{Technical Appendix}
%
% \section{Experimental Parameters} % possibly
%
% Again, I advise against using an appendix.

\end{document}

%  LocalWords:  LaTeX tex moretexcs Lübeck pdf uzl lualatex bibtex th
%  LocalWords:  TechReport Kernighan Lamport's Tantau's Tantau cls kZ
%  LocalWords:  Mustermann emacs oldschool pdflatex texmf utf biber
%  LocalWords:  biblatex Alphabetische Bibliographie Numerische VIIa
%  LocalWords:  varioref german Einleitung Beiträge dieser Arbeit xml
%  LocalWords:  Ergebnisse Verwandte Arbeiten Aufbau nucleotide VIIc
%  LocalWords:  ensembl amino phylogenetic Alexa Siri decrypt versa
%  LocalWords:  cryptographic pre nondeterministic deterministically
%  LocalWords:  Beutelspacher Untersuchungen zum genetischen sep llcc
%  LocalWords:  Beispiel tikz jpg png Alegrya Kasimir Malewitsch PGF
%  LocalWords:  Lamport Institut für Theoretische Informatik zu url
%  LocalWords:  Universität Springer DowneyF Downey Parameterized doi
%  LocalWords:  BibLaTeX Kime Philipp urldate Mittelbach hyperref Lua
%  LocalWords:  Rahtz Oberdiek Heiko Braams Bezos López fontspec Das
%  LocalWords:  Arseneau amsmath ist Tipps und zur Formulierung
%  LocalWords:  mathematischer Gedanken Mathematik Studienanfänger
%  LocalWords:  Albrecht Vieweg Teubner Verlag
