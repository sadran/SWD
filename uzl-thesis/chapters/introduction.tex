\chapter{Introduction}
% In a German thesis write: \chapter{Einleitung}


The increasing complexity of modern automotive systems has led to a growing
demand for precise in-cabin monitoring technologies. Among these, 
steering wheel detection plays a pivotal role in monitoring driver
behavior, enhancing safety, and facilitating autonomous control in 
advanced driver-assistance systems. Estimating the 3D location and 
orientation of the steering wheel is crucial for systems that rely on
driver interaction and vehicle control. 
Yet, achieving accurate 3D detection of such an object within a car's 
interior presents unique challenges, including variations in rotation and
position within confined, occlusion-prone environments. This is because 
the car's interior is a complex and constrained environment, with limited 
sensor coverage and frequent occlusions of key components like the 
steering wheel. Accurately determining the 3D position and orientation of 
the steering wheel is essential for enabling advanced driver monitoring 
and assistive technologies, but poses significant technical hurdles 
compared to object detection in open, outdoor environments. 


\section{Contributions of This Thesis}
This thesis addresses the problem of steering wheel detection using 3D 
point cloud, which offers a rich representation of spatial information. 
Specifically, this research contributes in two significant ways: 

First, it introduces a novel dataset that represents the car interior, 
including 3D point clouds and annotated bounding boxes for the steering 
wheel. This dataset was created through a meticulous process involving 
the recording of point cloud data, feature extraction, and preprocessing 
to ensure data integrity and clarity. By accurately determining the 
steering wheel’s 3D position and orientation, this dataset serves as a 
critical foundation for model training and validation.

Second, this thesis proposes a model-based approach for detecting the 
steering wheel’s 3D position using a modified VoxelRCNN architecture. 
Existing 3D object detection models, such as those trained on KITTI 
dataset data, are typically optimized for outdoor environments with 
non-cubic point cloud dimensions and restricted rotational degrees of 
freedom (DOF). To overcome these limitations, the research modifies the VoxelRCNN 
architecture to account for a cubic point cloud structure and allows for rotation detection along 
the x-axis, which is crucial for accurately representing the orientation of the steering wheel.

\section{Structure of This Thesis}
This thesis is organized to guide the reader through the development of a dataset and model for 3D steering wheel detection within the context of in-cabin monitoring for autonomous driving. Chapter 1, \textbf{Introduction}, presents the motivation for this research and highlights the specific contributions of this thesis. Chapter 2, \textbf{Related Work}, provides an overview of the existing literature on 3D object detection in autonomous driving, covering detection methods, commonly used datasets, and performance metrics, along with a summary of current challenges in the field.

Chapter 3, \textbf{Creating Dataset}, details the process of building a custom dataset for steering wheel detection. This includes data collection, preprocessing, ground truth generation, and the experiments conducted to validate the dataset. Both an initial and refined approach to ground truth generation are discussed, with results presented to evaluate the dataset’s effectiveness. Chapter 4, \textbf{Estimating Steering Wheel Position}, focuses on the architecture and adaptations made to a 3D object detection network for accurately detecting the steering wheel. It includes an evaluation of the model’s performance based on the developed dataset. The thesis concludes with Chapter 5, \textbf{Conclusion}, summarizing the key findings and suggesting directions for future research.



