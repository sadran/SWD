\chapter{Introduction}
% In a German thesis write: \chapter{Einleitung}


The increasing complexity of modern automotive systems has led to a growing
demand for precise in-cabin monitoring technologies. Among these, 
steering wheel detection plays a pivotal role in monitoring driver
behavior, enhancing safety, and facilitating autonomous control in 
advanced driver-assistance systems. Estimating the 3D location and 
orientation of the steering wheel is crucial for systems that rely on
driver interaction and vehicle control. 
Yet, achieving accurate 3D detection of such an object within a car's 
interior presents unique challenges, including variations in rotation and
position within confined, occlusion-prone environments. This is because 
the car's interior is a complex and constrained environment, with limited 
sensor coverage and frequent occlusions of key components like the 
steering wheel. Accurately determining the 3D position and orientation of 
the steering wheel is essential for enabling advanced driver monitoring 
and assistive technologies, but poses significant technical hurdles 
compared to object detection in open, outdoor environments. 


\section{Contributions of This Thesis}
This thesis addresses the problem of steering wheel detection using 3D 
point cloud, which offers a rich representation of spatial information. 
Specifically, this research contributes in two significant ways: 

First, it introduces a novel dataset that represents the car interior, 
including 3D point clouds and annotated bounding boxes for the steering 
wheel. This dataset was created through a meticulous process involving 
the recording of point cloud data, feature extraction, and preprocessing 
to ensure data integrity and clarity. By accurately determining the 
steering wheel’s 3D position and orientation, this dataset serves as a 
critical foundation for model training and validation.

Second, this thesis proposes a model-based approach for detecting the 
steering wheel’s 3D position using a modified VoxelRCNN architecture. 
Existing 3D object detection models, such as those trained on KITTI 
dataset data, are typically optimized for outdoor environments with 
non-cubic point cloud dimensions and restricted rotational degrees of 
freedom (DOF). To overcome these limitations, the research modifies the VoxelRCNN 
architecture to account for a cubic point cloud structure and allows for rotation detection along 
the x-axis, which is crucial for accurately representing the orientation of the steering wheel.

\section{Structure of This Thesis}
The structure of this thesis is organized to guide the reader through the progression from background research to dataset development, methodology, and results, concluding with a discussion of findings. Chapter 1 introduces the topic and outlines the key contributions of the thesis, setting the stage for the research conducted. Chapter 2 provides background information, covering essential concepts in 3D object detection for autonomous driving, reviewing related work, and summarizing existing datasets and performance metrics. Chapter 3 details the creation of the SWD dataset, describing the recording procedure, data extraction process, and ground truth generation. Chapters 4 and 5 focus on two approaches for generating ground truth: Chapter 4 describes the initial approach using circle fitting, while Chapter 5 explains the refined approach involving ArUco board including an introduction, methodology, evaluation, results, and conclusion sections. Chapter 6 presents the model and methodology used in training and evaluating the steering wheel detection model. Chapter 7 discusses the results and insights gained from the experiments, while Chapter 8 concludes the thesis by summarizing the findings, implications, and potential directions for future work. This structured approach ensures a comprehensive understanding of the methodologies and advancements made in estimating the steering wheel’s 3D position using point clouds.




