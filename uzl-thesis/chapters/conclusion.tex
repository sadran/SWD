\chapter{Conclusion}

This thesis presented a novel approach for detecting and localizing the steering wheel within 3D point clouds, specifically addressing the challenges of in-cabin automotive monitoring. The research contributed two primary innovations: the creation of the SWD dataset, capturing detailed 3D spatial representations of steering wheel positions, and the adaptation of the Voxel R-CNN model to meet the unique requirements of steering wheel detection. Together, these efforts have enabled precise and reliable estimation of both the 3D position and orientation of the steering wheel, a critical step for enhancing in-cabin monitoring and driver-assist systems.

The creation of the SWD dataset, with its tailored recording setup and structured marker arrangement, established a robust foundation for training and evaluation. The dataset’s comprehensive coverage across various angles, positions, and driver interactions provided a challenging yet realistic environment for testing detection capabilities in confined spaces. Through custom data preprocessing, such as distortion correction and noise reduction, the dataset facilitated precise ground truth generation, ensuring accuracy in evaluating the model’s spatial understanding.

Adapting Voxel R-CNN to this task involved several key modifications, including adjusting bounding box encoding to prioritize x-axis rotation and customizing the model’s 3D backbone to handle cubic point clouds effectively. These modifications addressed the specific spatial and rotational characteristics of the steering wheel, resulting in high recall and precision in detecting the steering wheel across 3D and Bird’s Eye View (BEV) projections. Evaluation metrics further validated the model’s performance, with high mAP scores, low distance error, and minimal rotation error confirming the model’s robustness in both positioning and orientation estimation.

The results of this research demonstrate the feasibility of deploying 3D object detection for in-cabin applications, particularly in scenarios where accurate steering wheel tracking is essential. The model’s ability to generalize across a variety of perspectives and orientations underscores its potential for practical use in automotive settings, contributing to safety and monitoring systems. The findings of this thesis not only highlight the adaptability of Voxel R-CNN but also lay a foundation for further advancements in 3D in-cabin monitoring.

In conclusion, this thesis provides a comprehensive approach to steering wheel detection using 3D point clouds, paving the way for enhanced in-cabin monitoring solutions. The advancements achieved here form a valuable basis for future research and application in autonomous driving and driver-assistive technology, where precise spatial awareness of in-cabin elements is increasingly vital.