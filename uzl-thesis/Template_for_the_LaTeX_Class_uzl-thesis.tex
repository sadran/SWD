\documentclass[english,version-2020-11]{uzl-thesis}


% Copy this file as a template for your thesis. You will have to take
% action at all places marked by
%
% !!!!!!!!!!!!!!!!!!!!!!!!!!!!!!!!!!
% !!! Your action is needed here !!!
% !!!!!!!!!!!!!!!!!!!!!!!!!!!!!!!!!!
%
% The first place your action is needed is the first line of this
% document:
%
%
% Language of the thesis:
%
% You must use either 'german' or 'english' above, depending on the
% language used in the main text. This will automatically setup a lot
% of things in the background.
%
%
% Version of the class:
%
% You must specify which version of the thesis class is to be
% used. This is important in case the class style changes in later
% years, but we still want an older thesis to look the same, even when
% things are changed in the class.
%
% Do not change or remove the version-xxxx key.
%
%
% Text encoding:
%
% Your thesis *must* be encoded in utf8 (unicode), which is the
% default in most editors these days. Do *not* change this to latin8.



%%%
%
% Main setup:
%
%%%
%
% You must use the \UzLThesisSetup command to specify numerous things
% about your thesis. This includes the entries on the title page, the 
% abstracts, and the bibliography style. You do so by specifying
% so-called "values" for so-called "keys". For instance, 
% for the key "Autor" you must provide your name as the value. You do
% so by writing 'Autor = {Max Mustermann}', that is, the value is put
% into curly braces. You can use the \UzLThesisSetup command
% repeatedly and the order in which you provide the keys is not
% important. 
%
% Everything shown on the title page must be in German -- even
% if the thesis is written in English! Just insert German text for
% German keys and English text for English keys (like 'Abstract' needs
% English text, while 'Zusammenfassung' needs German text).

\UzLThesisSetup{
  %
  % !!!!!!!!!!!!!!!!!!!!!!!!!!!!!!!!!!
  % !!! Your action is needed here !!!
  % !!!!!!!!!!!!!!!!!!!!!!!!!!!!!!!!!!
  %
  % First, specify the institut or clinic at which the thesis was
  % written. You get the logo file from them (make sure it has the
  % correct size, namely the same as the example). If they do not have
  % a logo, the university's default logo is used.
  %
  % The 'verfasst' gets two arguments. Change the first to {an der}
  % for clinics, as in 'Verfasst = {an der}{Medizinischen Klinik I}'
  %
  Logo-Dateiname        = {uzl-thesis-logo-itcs.pdf},
  Verfasst              = {am}{Institut für Theoretische Informatik},
  %
  % The titles:
  %
  Titel auf Deutsch     = {
    Vorlage für die \LaTeX-Klasse »uzl-thesis« zur Nutzung bei
    Bachelor-~und Masterarbeiten an der  Universität~zu~Lübeck
  }, 
  Titel auf Englisch    = {
    Template for the \LaTeX\ Class “uzl-thesis” for
    Bachelor's and Master's Theses Written at the University~of~Lübeck 
  },
  %
  % Author and supervisor:
  % 
  % Note that the 'Betreuer' or 'Betreuerin' is the supervisor, that
  % is, the professor who officially supervises the thesis. If there
  % is also an assistent of the professor who helped (typically a
  % lot), use 'Mit Unterstützung von' to thank that person. If the
  % thesis was mainly written 'externally' at some company or another
  % institute, point this out using 'Weitere Unterstützung'. 
  % 
  % For your own name, do *not* add things like "BSc" or "BSc
  % cand.". For the supervisor, you should normally include
  % "Prof. Dr." or "PD Dr." (ask your supervisor, what is
  % appropriate), but nothing more (so no
  % "Univ.-Prof. Dr. Dr. h.c. mult." unless your supervisor insists).  
  %
  Autor                 = {Max Mustermann (alias Till Tantau)},
  Betreuerin            = {Prof. Dr. Petra Wichtig-Wichtig},
  % 
  % Optional: Supporting persons and institutions. The text should be
  % in German, even for an English thesis.
  %
  Mit Unterstützung von = {Harry Hilfreich},
  % 
  %   Weitere Unterstützung = {
  %     Die Arbeit ist im Rahmen einer Tätigkeit bei der Firma Muster GmbH
  %     entstanden.
  %   },
  %
  %
  % Your Degree Programm (Studiengang)
  %
  % Specify 'Bachelorarbeit' or 'Masterarbeit' and the degree
  % programme. Make sure the name of programme is correct and not
  % some abbreviation or some incorrect variant. For instance:
  % 'Medizinische Ingenierwissenschaft', but not 'MIW';
  % 'Medizinische Informatik', but not 'Medizin-Informatik';
  % 'Informatik', but not 'Informatik (SSE)'.
  %
  % Use German names for German programmes and English names for
  % English ones, so 'Infection Biology', not 'Infektionsbiologie'. 
  % For programmes that have a German bachelor and an English master,
  % use the German name for a bachelor thesis and the English name for
  % the master thesis.
  %
  Bachelorarbeit,
  Studiengang           = {Informatik},
  %
  % Date on which the thesis is turned in German, formatted the
  % traditional German way:
  %
  Datum                 = {1. Januar 2021},
  %
  % The English abstract. You must always provide abstracts in German
  % and in English. 
  %
  Abstract              = {
    It is not easy to write a thesis that does not only advance
    science, but that is also a pleasure to read. While the scientific
    contribution of a thesis is undoubtedly of greater importance, the
    impact of \emph{writing well} should not be underestimated: If
    the person who grades a thesis finds no pleasure in the reading,
    that person are also unlikely to find pleasure in giving outstanding
    grades. A well-written text uses good German or English phrasing with a clear and correct 
    sentence structure and language rhythm, there are no spelling
    mistakes and the author's arguments are presented in a
    clear, logical and understandable manner using well-chosen
    examples and explanations. In addition, a nice-to-read font and a
    pleasing layout are also helpful. The \LaTeX\ class presented in
    this document helps with the latter: It contains a number of
    ready-to-use designs and 
    takes care of many small typographical chores.
  },
  Zusammenfassung       = {
    Es ist nicht leicht, eine Abschlussarbeit so zu schreiben, dass sie
    nicht nur inhaltlich gut ist, sondern es auch eine Freude ist, sie
    zu lesen. Diese Freude ist aber wichtig: Wenn die Person, die die 
    Arbeit benoten soll, wenig Gefallen am Lesen der Arbeit findet,
    so wird sie auch wenig Gefallen an einer guten Note
    finden. Glücklicherweise gibt es einige Kniffe, gut lesbare
    Arbeiten zu schreiben. Am wichtigsten ist zweifelsohne, dass
    die Arbeit in gutem Deutsch oder Englisch verfasst wurde mit klarem
    Satzbau und gutem Sprachrhythmus, dass keine Rechtschreib- oder
    Grammatikfehlern im Text auftauchen und dass die Argumente der
    Autorin oder des Autors klar, logisch, verständlich und gut
    veranschaulicht dargestellt werden. Daneben sind aber auch gut
    lesbare Schriftbilder und ein angenehmes Layout hilfreich. Die Nutzung
    dieser \LaTeX-Vorlage hilft der Schreiberin oder dem Schreiber
    dabei zumindest bei Letzterem: Sie umfasst gute, sofort nutzbare
    Designs und sie kümmert sich um viele typographische
    Details.  
  },
  %
  % Optional: 'Danksagungen' (German) or 'Acknowledgements'
  % (English). Both keys are optional and both have the same effect of
  % adding an acknowledgements text after the abstracts and before the
  % table of contents.
  %
  Acknowledgements      = {
    This is the place where you can thank people and institutions, do
    not try to do this on the title page. The only exception is in
    case you wrote your thesis while working or staying at a company or abroad. Then you
    should use the \Latex{Weitere Unterstützung} key to provide a text
    (in German) that acknowledges the company or foreign
    institute. For instance, you could use texts like »Die Arbeit
      ist im Rahmen einer Tätigkeit bei der Firma Muster GmbH
      entstanden« or »Die Arbeit ist im Rahmen eines
      Forschungsaufenthalts beim Institut für Dieses und Jenes an der
      Universität Entenhausen entstanden«. Do not name and thank
      individual persons from the company or foreign institute on the
      title page, do that here. 
  },
  % Bibliography style: Choose between
  % 
  % 'Alphabetische Bibliographie'
  % for all degree programmes in the natural sciences 
  % 
  % 'Numerische Bibliographie'
  % alternative for all other degree programmes
  % 
  % Either will load biblatex and setup the citation methods and the
  % bibliography styles correctly. You should not mess with them.
  % 
  Alphabetische Bibliographie,
  % Alternatively:
  % Numerische Bibliographie
}




%%%%%%%%%%%%%%%%%%%%
%
% Styling the thesis
%
%%%%%%%%%%%%%%%%%%%%
%
% Creating a visually pleasing layout and choosing fonts is not
% easy. Furthermore, different people have different preferences. Of
% course, for the University of Lübeck, the dean of studies could just
% force everyone to use one specific layout and font, but that seems a
% bit drastic and, also, it seems nice that thesis by different people
% have an individual style even though they all stick to the same
% overall structure.
%
% For these reasons, I (Till Tantau) have spend quite some time on
% designing a flexible layout and styling mechanism for theses.
%
% Basically, the overall structure of the thesis is fixed by the
% thesis class and so are many structural elements. For instance, you
% cannot change the order in which the abstract and table of contents
% are shown, you cannot move the bibliography elsewhere, indeed, the
% bibliography style is also fixed. Likewise, the text on the title
% page is fixed.
%
% Although many things are fixed, you *can* change several other
% things. For instance, you can change the font used for the main
% text, you can change which font is used for titles and headings or
% you can change whether titles and headlines are centered or flushed
% left.
%
% There are many LaTeX packages for changing such things. You are
% kindly asked *not to use them*. Rather, use (only) the options
% offered by the thesis class. All possible choices and combinations
% there have been tested by me and produce nice results; what happens
% with other packages no one knows and might no longer conform to what
% is expected by the university. As you will see, you still have a
% lot of options.
%
%
% Technical note: All styling is done via the command
%
% \UzLStyle{...}
%
% where ... is a key-value list just as for \UzLThesisSetup. The
% difference is just that everything having to do with styling as
% controlled by \UzLStyle, while the more “formal” setup keys are
% controlled by \UzLThesisSetup.
%
%%%
%
% Designs
%
%
% A \emph{design} is a whole set of font and layout options bundled
% together. They have been chosen in such a way that a visually
% pleasing “overall appearance” results.
%
%
% \UzLStyle{computer modern oldschool design}
%
% The look of this design mimics the “classical” way a paper or report
% created with \LaTeX\ looks like: The Computer Modern font is used,
% bold face fonts are used for headlines, only black and white are
% used as colors. This design reminds me of older scientific
% documents, especially from the computer science community where
% \LaTeX\ was used very early.
%
%
% \UzLStyle{computer modern basic design}
%
% A slightly less “oldschool” version of the previous design. It is
% still a classic design in the sense that it uses the Computer Modern
% font and that it still has this “good old \LaTeX” look, but some
% more modern aspects (like colors!) have been added.
%
% Note that this design uses Myriad for the title page (one of the
% “modern aspect”), which means that his font must be installed.
%
%
% \UzLStyle{computer modern scholary design}
%
% In my opinion, this is the ultimate “scholary design”: The thesis
% will look like it has been typeset by hand some 150 years ago and
% then printed by a university press. There is really nothing “modern”
% about it and the word in the name of the design is just part of the
% name of the “Computer Modern” font.
%
%
% \UzLStyle{pagella basic design}
%
% A, well, basic design that uses the Pagella font rather than the
% Computer Modern font. Especially the bold face version of this font
% looks nicer than the Computer Modern counterpart. Also, Pagella,
% while still having a “bookish” look, still feels a bit fresher than
% Computer Modern. 
%
%
% \UzLStyle{pagella centered design}
%
% A variant of the basic Pagella design that centers all
% headlines. A nice alternative to the basic version.
%
%
% \UzLStyle{pagella contrast design}
%
% This design tries to create some visual friction by contrasting the
% sans serif headline font (in bold!) with the main text. I find it a
% visually very interesting combination.
%
%
% \UzLStyle{alegrya basic design}
%
% The third variant of the basic design, this time using the Alegrya
% font. 
%
%
% \UzLStyle{alegrya scholary design}
%
% The Alegrya version of the previous “scholary” design. Unlike the
% Computer Modern version, this design does not look old, but more
% fresh -- while still creating the impression that the text must be
% about a very scientific subject. 
%
%
% \UzLStyle{alegrya stylish design}
%
% The design is quite similar to the scholary version for the Alegrya
% font, but with even more modern additions. “Stylish” is the word
% that comes to my mind.
%
%
\UzLStyle{alegrya modern design}
%
% A design that uses the sans serif version of the Alegrya font for
% the headlines. This is a nice modern overall design.
%
%%%




%%%%%%%%
%
% Now, include the package you need here using \usepackage. 
%
% However, many standard packages are already loaded by the class:
%
% amsmath, amssymb, amsthm, babel, biblatex, csquotes, etoolbox,
% filecontents, fontspec, geometry, hyperref, tikz (with libraries
% arrows.meta, positioning and shapes), varioref, url 
%
% Indeed, in many cases you will not need any extra packages.
%
%%%%%%%





\begin{document}

%
% The title page and table of contents will be inserted automatically
% here. 
%


\chapter{Introduction}
% In a German thesis write: \chapter{Einleitung}


% !!!!!!!!!!!!!!!!!!!!!!!!!!!!!!!!!!
% !!! Your action is needed here !!!
% !!!!!!!!!!!!!!!!!!!!!!!!!!!!!!!!!!
%
% Replace with your own introduction:

% DELETE the following line for your own thesis - it causes trouble!
\lstMakeShortInline[style=code,style=inline,language={[LaTeX]tex},moretexcs={chapter}]|

Writing a bachelor's or master's thesis is not easy.\footnote{Neither
  is writing a PhD thesis, but this document does \emph{not} concern
  them. It is intended \emph{only} as a template for bachelor's and
  master's theses written at the University of Lübeck. When you write
  a PhD thesis, you are invited to find your own style.} You must
\emph{research} the thesis's topic scientifically 
-- and you must do 
this well. You must \emph{describe} your research and the results --
and you must use not just any words, but those that are used in the
scientific community. Finally, you must \emph{write} everything
\emph{down} -- by creating an electronic document that is a pleasure
to read.

It is the last item where this text may help: It is, first, a
\emph{template} that you, dear student, can copy and then modify when
writing your thesis. Of course, you still have to write the text, but
the template will take care of numerous technical details for you. As
a teaser, have a look at Listing~\vref{listing-hello-world}, which 
shows the code for the ``hello world of theses''\footnote{In computer
  science, a ``Hello World'' program is a minimal program in a given
  programming language that   just prints these two words.} and which
already produces a \textsc{pdf} file with five pages of so-called front
matter (like the title page, the abstract or the table of contents)
and already four pages of actual content -- not bad for a single page
of code.


\begin{Latex}[
  float,
  caption={
    Minimal \LaTeX\ manuscript that generates a bachelor's thesis using the
    \textsc{uzl-thesis} class. The manuscript has to be processed twice
    using \Latex{lualatex}, followed by a run of \Latex{bibtex},
    followed by a run of \Latex{lualatex} once more. 
  },
  label={listing-hello-world}
  ]
\documentclass[english, version-2020-11]{uzl-thesis}

\UzLStyle{computer modern oldschool design}

\UzLThesisSetup{
  Bachelorarbeit,
  Verfasst = {am}{Institut für tolle Forschung},
  Titel auf Deutsch = {Hallo Welt},
  Titel auf Englisch = {Hello World},
  Autor = {Max Mustermann},
  Betreuerin = {Prof. Dr. Petra Wichtig-Wichtig},
  Studiengang = {Irgendwas mit Tieren},
  Datum = {1. Juli 2020},
  Abstract = {It is about saying ``hello'' to the world.},
  Zusammenfassung = {Es geht darum, der Welt »Hallo« zu sagen.},
  Numerische Bibliographie
}

\begin{document}
  \chapter{Introduction}
  \section{Contributions of this Thesis}
  This thesis says ``Hello World!'', see also \cite{Kernighan1974}.
  \section{Related Work}
  There are many hello world programs.
  \section{Structure of this Thesis}
  In Chapter~\vref{chapter-main}, we say hello.
  
  \chapter{Main Chapter}
  \label{chapter-main}
  Hello World!

  \chapter{Conclusion}
  Saying hello world is quite easy.

  \begin{bibtex-entries}
    @TechReport{Kernighan1974,
      author = {Brian Kernighan},
      title = {Programming in C – A Tutorial},
      institution = {Bell Laboratories},
      year = {1974}
    }
  \end{bibtex-entries}
\end{document}
\end{Latex}

This template document is a \LaTeX\ document that uses the
\textsc{uzl-thesis} document class. This means that in order to work with it,
you need to use Donald Knuth's \TeX\ text processing system
\cite{Knuth1986}, Leslie Lamport's \LaTeX\ extension of \TeX\
\cite{Lamport1994} and my
(that is, Till Tantau's) 
\textsc{uzl-thesis} document class. In particular, you will need to learn
\LaTeX\ if you have not already done so (definitely a good idea
anyway). 

Some students may wonder at this point whether this text
applies to them at all since they do not intend to (or perhaps even
may not) use \LaTeX\ for their thesis. However, while these readers
can safely skip the technical details of how the \textsc{uzl-thesis} class is
used, I would like to urge them (and, of course, everyone else) to read
Chapter~\ref{chapter-tips}, starting \vpageref{chapter-tips}: In this
chapter, I explain my views on 
``how to write a good thesis'' and try to give as many practical hints
as possible that anyone attempting to write a thesis will hopefully
find useful -- independently of which text processing tool they use.

The ``hints'' given in Chapter~\ref{chapter-tips} address many of the
problems that I see students struggle with when they write their
thesis. Of course, I cannot give a magic recipe for creating a
scientific breakthrough. But I \emph{can} give you hints on how to
put a breakthrough into words that other people understand and will
like to read -- and, hopefully, will like to reward with good grades.

Please be aware that the views expressed in Chapter~\ref{chapter-tips}
are \emph{my} views and some of them may not be shared by other
professors and, more importantly for you, they may not be shared by
your adviser -- who happens to be the person who will grade your
thesis. This means that you better \emph{always listen to your
  adviser} and do what she or he asks you to do.\footnote{If your adviser
thinks the thesis should be typeset using a typewriter font with
double line spacing and all headlines should be in pink, then I may
(very) strongly disagree with that, but you do not have that luxury
and you just typeset everything in double line spacing pink
typewriter.} The excuse ``but 
Professor Tantau writes that\dots'' may be flattering to me, but it
will not get you high grades.

So, \emph{always listen to your adviser.} You will read this again
later on. Repeatedly. 


\section{Contributions of this Thesis}
% In a German thesis write: \section{Beiträge dieser Arbeit}

% !!!!!!!!!!!!!!!!!!!!!!!!!!!!!!!!!!
% !!! Your action is needed here !!!
% !!!!!!!!!!!!!!!!!!!!!!!!!!!!!!!!!!
%
% Replace with a detailed account of your contributions:

This document is both intended as a template that students can easily
and conveniently adapt to their own needs and as a trove of tips and
tricks on how to write a good thesis. The template as well as the
\textsc{uzl-thesis} class have been written by Till Tantau and
they reflect -- to a certain degree -- my personal choices and many of
the recommendations may differ from those by other people. However,
one recommendation of mine trumps those made by anyone else:
\emph{Always listen to your adviser!}


\section{Related Work}
% In a German thesis write: \section{Verwandte Arbeiten}

% !!!!!!!!!!!!!!!!!!!!!!!!!!!!!!!!!!
% !!! Your action is needed here !!!
% !!!!!!!!!!!!!!!!!!!!!!!!!!!!!!!!!!
%
% Replace with a detailed account of what other people have already
% researched concerning your thesis's theme. Even when (indeed,
% especially when) there has been only little or even no research by
% other people, you should explain in detail that this is the case and
% why it is the case. 

Concerning the first aspect of this text, namely the fact that it
serves as a template for bachelor's and master's theses written at the
University of Lübeck, let me remark that there are several older
templates and styles. However, this class is the official (new) one
and should be used rather than any other one.

Concerning the set of recommendations I make on how to write a
thesis, first note that there are \emph{many} books on,
well, ``how to write a thesis''. You are \emph{very} cordially invited
to go to the library and actually read one of them (reading books
\emph{is} part of being a scientist, by the way). My personal favorite
is \citetitle{Alley1996} \cite{Alley1996}. Second, note that
the hints and observations I make are \emph{my} observations and may
not be shared by everyone. Of course, I can make a very good case on
why you really should follow the recommendations \emph{I} make -- but
remember that you should \emph{always listen to your adviser!}

\section{Structure of this Thesis}
% In a German thesis write: \section{Aufbau dieser Arbeit}


% !!!!!!!!!!!!!!!!!!!!!!!!!!!!!!!!!!
% !!! Your action is needed here !!!
% !!!!!!!!!!!!!!!!!!!!!!!!!!!!!!!!!!
%
% Replace the following by one or two paragraphs describing the
% thesis's structure.

This thesis\footnote{Actually, ``this text'' would be more appropriate
  since this is obviously not a real thesis. But this is what you
  would write in a real thesis at this point.} consists of two main
chapters: Chapter~\ref{chapter-use} describes how the
\textsc{uzl-thesis} \LaTeX\ class is used on a technical level. This chapter starts with the
technical details of how you setup the \TeX\ work-flow in conjunction
with the class (where to install it and which programs to use), but
the bulk of the chapter is taken up by the different aspects of using
that class -- like how bibliographies are created or how math text
should be written. The explanations only try to highlight what is
important and different when using the \textsc{uzl-thesis} class; they
are not intended as a complete introduction to \LaTeX. In
Chapter~\ref{chapter-tips}, I then list the many small and big things
you should consider and take care of when writing a thesis. I will
explain how long the different parts should be, I will sketch why the
abstract, the introduction and the conclusion all summarize the main
part of the thesis, but still all three need to be written, I will
explain why you should write ``we will show that'' and not ``I will
show that'' but ``I believe that'' and not ``we believe that'' and I
will give recommendations on many other topics. But of course,
whatever you read in the following, remember that you must
\emph{always listen to your adviser!}  








% !!!!!!!!!!!!!!!!!!!!!!!!!!!!!!!!!!
% !!! Your action is needed here !!!
% !!!!!!!!!!!!!!!!!!!!!!!!!!!!!!!!!!
%
% Replace the whole text chapter with the main text of your thesis! 

\chapter{Guide to the Thesis Class}%: Document Setup and Document Structure}
\label{chapter-use}


In this document, the \emph{thesis class} refers to the \LaTeX\
document class \textsc{uzl-thesis} (the file |uzl-thesis.cls| to be
precise), which was written by Till Tantau (that would be me, despite
the “Max Mustermann” from the title page) to help  
students of the University of Lübeck when they write their bachelor's
or master's thesis. In \LaTeX-speak, a \emph{document class} dictates
the basic appearance of a document and this is exactly what the thesis
class does: it sets up the appearance of the document in such a way
that it conforms (perfectly) with the regulations of the University of
Lübeck.

In addition to just formatting a thesis correctly, the thesis class
also provides a number of extra commands and includes a number of
packages that I have decided should be the standard at the
university. This means that most students will need to load very
few (typically no) extra packages for their thesis.

In this chapter, I explain all the “public” parts of the thesis class,
that is, all those aspects and configuration options that students can
and should use. There are also some internal keys and commands that
an adventurous student might try to fool around with. Don't. 




\section{Installing and Setting Up the System}

In order to write your thesis using the thesis class, you first need
to setup things on your system appropriately. With a bit of luck,
things have already been setup correctly for you, otherwise you will
need to follow the following steps:

First, since the thesis class is based on \LaTeX, you need a working
\LaTeX\ installation on the system on which you write your thesis. This
“system” can be a laptop or one of the university's computers or even
a cloud-based service; whatever it is, \TeX\ needs to be installed on
it. This is done using so-called \emph{distributions} and they depend
on the operating system and the details of your system. With a bit of
luck, you will already have a full distribution installed, otherwise
you need to get it from the net and install it. This is usually a
rather simple process, but it may take a while (a full installation
needs several gigabytes). I make no recommendation on which \TeX\
distribution to use, but usually ``simpler is better'' and any
contribution should be based on a recent version (at least 2015) of
\TeX live. I have tested the class extensively with the \TeX\
installation at the university computer labs, with \TeX live 2015 and
later, with Mik\TeX\ and with Debian-based Linux distributions
(where you may need to install the additional \TeX\ font package,
though). 

Second, you will need an editor for writing the text of the thesis
(sometimes this text is called the \emph{manuscript;} it is the file
that \TeX\ processes and turns into the final \textsc{pdf} file). This
editor needs to be a simple editor that is well-suited for handling
\TeX\ files. Once more, I make no recommendations on which editor you
should use: I use \textsc{emacs}, but that is somewhat oldschool. Use
any editor you feel comfortable with.

Third, you will actually have to run \TeX\ on your manuscript. How
this is done, exactly, will be discussed in the next section, but at
this point it is important that by ``running \TeX'' I actually mean
``running the |lualatex| program'': There are several \LaTeX\
programs out there, such as the original |latex|, the more
recent |pdflatex| and the even more recent |lualatex|. For the thesis
class, you \emph{must} use |lualatex|. While this program will almost
certainly  be already 
installed as part of the \TeX\ distribution you use, you may have to
change the settings of the editor you use so that it runs |lualatex|
rather than |pdflatex|.

It is well worth investing the time for this change, not only because
of the \textsc{uzl-thesis} class, but you should switch to |lualatex| in
general if you have not yet done so.   

Fourth and finally, you need a copy of the file |uzl-thesis.cls|
together with some logo \textsc{pdf} files. Since you are currently
reading this document, which is part of the directory containing the
class file, there is a very good chance that you already have all the
files you need. \emph{But} you may still need to place them somewhere
where \TeX\ will find them. A good place for this is usually the
directory that also contains your thesis manuscript (remember, that is
the file that contains your thesis and ends~“|.tex|”).\footnote{If you
  are more \TeX\ savvy, you may place the directory with the
  \Latex{uzl-thesis.cls} in a local \textsc{texmf} tree. However, in the
  long run, it is advisable to keep all files needed for typesetting
  you thesis together -- namely in the directory of the manuscript.}
I have carefully tried to make sure that the number of files needed
for the thesis class is as small as possible and that it should work
out-of-the-box on most systems.

Test whether everything works by running Lua\LaTeX\ on this file, that
is, by running:
\begin{Code}
lualatex Template_for_the_LaTeX_Class_uzl-thesis.tex
\end{Code}
If there is a problem or error, do not be too shy to ask somebody to
help you with the setup!  


\section{Running \LaTeX\ and Bib\TeX}

With the system up and running, you can start writing the
manuscript. The work-flow is the following: 


\begin{enumerate}
\item Make a copy of |Template_for_the_LaTeX_Class_uzl-thesis.tex| and
  rename the copy to 
  something like, for instance, |mythesis.tex|.\footnote{Note that
    the file is encoded in \textsc{utf8} encoding (unless you study
    computer science, you really do not wish to know the details) and
    that the encoding should stay that way. Fortunately, most editors
    will use this encoding automatically, so this should not be a
    problem.} This will be the \emph{manuscript} of your thesis.
\item
  Replace all parts of the text (like this one) with your own
  text. Use your favorite editor for this.
\item
  Now run |lualatex| on the manuscript.
\item
  If there are \emph{any errors,} then \emph{correct them first.} It
  is usually a very bad idea to ignore the errors. Ask someone if you
  do not know how to fix an error! 
\item
  After you have successfully run \LaTeX\ at least twice, run |bibtex|. Once
  more, fix any errors before proceeding.\footnote{There is also the
    option of using \Latex{biber}, see Section~\vref{section-references} for 
    details. If you have no idea what \Latex{biber} is, just ignore this
    remark.} 
\item
  Then run |lualatex| once more. 
\item
  You should now have a correct \textsc{pdf} file.
\end{enumerate}


Congratulations on creating a first version of your thesis. As you may
already have noticed when looking at
Listing~\vref{listing-hello-world}, the thesis manuscript 
contains a rather prominent block in the preamble where the
|\UzLThesisSetup| command is used. In the following, we have a closer
look at this command; but, first, we have a look at the very first
line of the listing, namely the class options.


\section{Specifying the Language and Version}

Your manuscript must start with

\begin{Latex}
\documentclass[german, version-2020-11]{uzl-thesis}
\end{Latex}
%
or with
%
\begin{Latex}
\documentclass[english, version-2020-11]{uzl-thesis}
\end{Latex}
%
depending, of course, on whether your thesis is written in German or
in English. Note that no other languages are supported (or allowed)
and that you \emph{must} specify \emph{exactly} one of the two
language options\footnote{Do not worry about that fact than an English
  thesis will contain a German abstract text, which needs special
  hyphentation -- this is taken care of automatically. As I promised, 
  the thesis class takes care of many technical details that you may
  not have thought about.}.

You \emph{must} also provide the class version key\footnote{In
  computer science parlor, a “key” is a text like \Latex{version-2020-11}
  that is specified in lists and that causes some configuration to be
  chosen.} and note that this is the version and date of the thesis
  \emph{class} and not the date when you turn in \emph{your} thesis. The key ensures  
that when the thesis class is changed in the coming years, using a
new version together with your (original) manuscript will still produce
the same \textsc{pdf} as it did when you wrote your thesis. Just leave
this class version key as it is. 

\section{Specifying the Overall Design}

\label{section-styling}

The way a thesis looks -- its \emph{style} or \emph{design} -- is a
matter of taste; but only to a certain degree. While there are many
things that are “not open to discussion” like the fact that the main
font should have a certain minimum (and maximum) size and should be
easy to read, there is no reason why any \emph{particular} font
\emph{must} be used in a thesis or why a, say, centered layout is
better than a flushed-left layout.

With the thesis class, you can quite easily change the way the thesis
looks -- but you are limited to those choices that I have vetted and
that I know work well together. While this is definitely a better
approach than the one I see elsewhere (where each student copies
layout code from someone else, neither understanding what it does nor
what it is actually for, and then hopes for the best), I restrict your freedom when it
comes to styling the thesis: Even if you really think that pink
headlines and a slab serif as main font are way cooler that the
designs I chose, you will just have to live with the options I
provide. I guess that 99\% of all students will be more than happy to
leave the design process to me, but if you are one of the 1\%, you may
complain fiercely at this point and may then just pick one of the
styles I chose for you like everyone else.


\subsection{Choosing a Design}

To pick a design, you call one command in the preamble (that is
the part following the |\documentclass| before the actual thesis
starts with the |{document}| environment):
\begin{Latex}[style=escapemath]
\UzLStyle{$\langle \mathit{style}\rangle$}
\end{Latex}
where $\langle \mathit{style}\rangle$ is one of the following (all in
lowercase, as in |\UzLStyle{alegrya basic design}| for instnace):
\begin{description}
\item[computer modern oldschool design] 
  The look of this design mimics the “classical” way a paper or report
  created with \LaTeX\ looks like: The Computer Modern font is used,
  bold face fonts are used for headlines, only black and white are
  used instead of colors. This design reminds me of older scientific
  documents, especially from the computer science community where
  \LaTeX\ was used very early.
\item[computer modern basic design]
  A slightly less “oldschool” version of the previous design. It is
  still a classic design in the sense that it uses the Computer Modern
  font and that it still has this “good old \LaTeX” look, but some
  more modern aspects (like colors!) have been added.
  
  Note that this design uses the Myriad font for the title page (one
  of the “modern aspects”), which means that this font must be
  installed. This font is the university's official font, which is why
  it is appropriate for the title page. However, the font does not otherwise
  go well with the Computer Modern font, so it is used only on the
  title page.
\item[computer modern scholary design]
  In my opinion, this is the ultimate “scholary design”: The thesis
  will look like it had been typeset by hand some 150 years ago and
  then printed by a university press. There is really nothing “modern”
  about it despite this word being part of the design's name: the word
  \emph{modern} is just part of the name of the Computer \emph{Modern} font. 
\item[pagella basic design]
  A, well, basic design that uses the Pagella font rather than the
  Computer Modern font. Especially the bold face version of this font
  looks nicer than the Computer Modern counterpart. Also, Pagella,
  while still having a “bookish” look, feels a bit fresher than
  Computer Modern. 
\item[pagella centered design]
  A variant of the basic Pagella design that centers all
  headlines. A nice alternative to the basic version.
\item[pagella contrast design]
  This design tries to create some visual friction by contrasting the
  sans serif headline font (in bold!) with the main text. I find it a
  visually very interesting combination.
\item[alegrya basic design]
  The third variant of the basic design, this time using the Alegrya
  font. 
\item[alegrya scholary design]
  The Alegrya version of the previous “scholary” design. Unlike the
  Computer Modern version, this design does not look old, but more
  fresh -- while still creating the impression that the text must be
  about a very scientific subject. 
\item[alegrya stylish design]
  The design is quite similar to the scholary version for the Alegrya
  font, but with even more modern additions. “Stylish” is the word
  that comes to my mind. 
\item[alegrya modern design]
  A design that uses the sans serif version of the Alegrya font for
  the headlines. This is a nice modern overall design.
\end{description}


\subsection{Notes on Fonts}

The different designs from which you can choose use three different
main fonts: Computer Modern, Pagella and Alegrya. Of course, there are
many more fonts around and even installed on your computer -- but you
can still use only one of these three. The reason is that there is
much more to choosing and setting up a font than meets the eye: One
must ensure that the font works not only for simple text, but also for
complex mathematical text, that the font sizes fit and are comparable
to those of other fonts, that the typewriter (the monospace) version
goes well with the main font, that the small caps version is correctly
selected and much more.

All of these complex setups have been carefully done for these three
fonts (the setup for Alegrya alone needs 50 lines of code of the
thesis class). Since it is almost impossible for a beginner (or even
an advanced user) to get all of these setups right when choosing
another font, you are kindly requested to refrain from doing that. Use
one of the three fonts, which get installed as part of the different
designs. 


\subsection{Further Styling}

The styling mechanism actually allows you finer control over the
styling of different elements of a thesis, like the font and color
used for section titles, the layout of the table of contents and many
more aspects. However, while you find some documentation in the
comments inside the thesis class file, I do not provide a detailed
documentation here since, usually, you should not configure these styles
individually. 


\section{Specifying the Metadata Like Title and Abstract}

You \emph{must} use the command |\UzLThesisSetup| (at least) once in the
preamble. This command is used to configure the so-called
\emph{metadata} prior to the first chapter of the thesis. The metadata
is information “about” your thesis (hence the “meta-”) and must be
provided by this special command. Note that there are normally other
commands in \LaTeX\ for providing such metadata (like |\author|), but
you \emph{may not use them} since all metadata is specified using the
|\UzLThesisSetup| command. The syntax is as follows: 
\begin{Latex}[style=escapemath]
\UzLStyle{$\langle \mathit{key}\rangle$ = {$\langle \mathit{value}\rangle$},
          $\langle \mathit{key}\rangle$ = {$\langle \mathit{value}\rangle$},
          $\dots$,
          $\langle \mathit{key}\rangle$ = {$\langle \mathit{value}\rangle$}}
\end{Latex}
%
For some keys, the |= {|$\langle \mathit{value}\rangle$|}| part may be
missing. You can use the command several times, but all uses must be
within the preamble and usually you will put everything in a single
command. There are numerous keys that can be configured with this
command and most of them are required, that is, they \emph{must} be
set (exactly once). The order in which you set them is, however, not
important.

The keys that you \emph{must} set are the following (note that most of
them are in German; this is to remind you that you \emph{must} provide
German texts as values for these keys \emph{even in an English thesis}
since the university's guidelines insist that the text on the title
page must be in German):

\begin{description}
\item[Bachelorarbeit / Masterarbeit] You must
  specify exactly one of these two keys. Note that this thesis class
  is \emph{not} meant for PhD theses.
\item[Verfasst] Specify the institute or clinic where you
  wrote the thesis. Note that this must be the official name of one of
  the institutes and clinics of the university (sometimes even the
  supervisors do not seem to know the real name of their own
  institute\dots). The key takes two arguments as value (it is a
  rather special key): The first is the word |am| (for institutes) or
  the words |an der| (for clinics), the second is the name of the
  institute or clinic. Here are two examples:
\begin{Latex}
  Verfasst = {am}{Institut für Theoretische Informatik}
\end{Latex}
  or
\begin{Latex}
  Verfasst = {an der}{Klinik für Hals-, Nasen- und Ohrenheilkunde}
\end{Latex}
  It would be wrong to use “\foreignlanguage{german}{Klinik für HNO}”
  instead. If in doubt, 
  consult the “\foreignlanguage{german}{Sat\-zung über die Institute
    und Kliniken der Universität zu Lübeck}” from the university
  homepage. It would be \emph{utterly wrong} to write
  “\foreignlanguage{german}{Universitätsklinikum 
  Schleswig-Hol\-stein}” here, by the way, since that institution is not
  part of the university.

  Note that if you have written your thesis externally (like abroad or
  at a company), you \emph{still} have a supervisor at the University
  of Lübeck and she or he is \emph{still} affiliated with an institute
  of the university. It is \emph{that} institute or clinic that must
  be named here.
\item[Logo-Dateiname] If the institute has provided you with its own
  logo as a \textsc{pdf} file, you can specify the file name
  here. Note that only those logos are allowed here that look like the 
  university logo (see Figure~\vref{fig-logo}), but with the name of
  the institute added. Also note that the logo \textsc{pdf} file
  should have the exact same size and scaling as the normal logo.
\item[Titel auf Deutsch] Provide the German title here. \emph{Please}
  try to avoid spelling mistakes at least here. Wrong capitalization 
  \emph{is} a spelling mistake in German, by the way: Do \emph{not}
  entitle your thesis “\foreignlanguage{german}{Untersuchungen zur Schnellen Simuluation von
  Turing Maschinen}” -- that is two spelling mistakes! You must use
  “\foreignlanguage{german}{Untersuchungen zur schnellen Simulation von Turing-Maschinen}”.  
\item[Titel auf English] Provide the English title here. Note that
  there are \emph{also} rules for capitalization in English, but they
  are a bit more flexible. For more information on when to use
  uppercase and when to use lowercase in English, see
  Question~\vref{qu-titles}. 
\item[Studiengang] Make sure the name of degree programme is
  correct and not some abbreviation or some incorrect variant. For
  instance: “\foreignlanguage{german}{Medizinische Ingenieurwissenschaft}”, but not “MIW”;
  “\foreignlanguage{german}{Medizinische Informatik}”, but not “\foreignlanguage{german}{Medizin-Informatik}”;
  “Informatik”, but not “Informatik (SSE)”. 
  
  Use German names for German programmes and English names for
  English ones, so “Infection Biology”, not “\foreignlanguage{german}{Infektionsbiologie}” and
  also neither “Infection-Biology” nor  “Infection biology”. 
  For programmes that have a German bachelor programme and an English
  master programme, use the German name for a bachelor's thesis and
  the English name for the master's thesis.  
\item[Autor / Autorin] Both keys have exactly the same effect. In
  either case, that would be you, I sincerely hope. Make sure
  you spell your own name correctly. It is slightly embarrassing to
  get it wrong (I saw that on a thesis, once; so, you have been
  warned). If you have more than one first name, you can include
  them.

  For your own name, do not add things like “BSc” or “BSc cand.”\
  (especially the latter is slightly ridiculous). For the adviser,
  you should normally include “Prof.\ Dr.”\ or “PD Dr.”\ (ask your
  adviser, what is appropriate), but nothing more (so no
  “Univ.-Prof.\ Dr.\ Dr.~h.\,c.~mult.”\ unless your adviser insists).  
\item[Betreuer / Betreuerin] Both keys have exactly
  the same effect. Note that this key specifies the \emph{adviser,}
  that is, the professor who officially supervised the thesis. If
  there is also an assistent of the professor who helped (typically a
  lot), use the key |Mit Unterstützung von| (see below) to thank that
  person. If the thesis was mainly written externally at some
  company or another institute, point this out using
  |Weitere Unterstützung| -- but note that the adviser will
  \emph{still} officially be a professor of the University of Lübeck
  and you must name her or him, here.
\item[Datum] The date you turn in your thesis. Do \emph{not} be
  late. Format the data in the “traditional German way“, that is, the
  first of June 2020 must be formatted as “1. Juni 2020“. Do
  \emph{not} write any of “01. Juni 2020” or “01. 06. 2020“ or
  “1.6.2020” or “1. 6. 2020” or “2020-06-01”.
\item[Abstract] Provide the (whole) English abstract of your thesis
  here. It may have more than one paragraph, but it should not. You will find
  more information on how to write an abstract in the answers to
  Questions \ref{qu-abstract1} and~\vref{qu-abstract2}.
\item[Zusammenfassung] Provide the German abstract here. It should be
  a faithful, but not word-to-word translation of the English one (or
  \emph{vice versa}). 
\item[Alphabetische Bibliographie /]
\item[Numerische Bibliographie] You
  must use exactly one of these two keys. They select which
  bibliography style is used throughout the thesis. If you write a
  thesis in a degree programme of the natural sciences (like Molecular
  Life Science), you \emph{must} use the first key. Otherwise, you can
  use either one. You can find more information on the bibliographies
  in Section~\vref{section-bibliographies}.
\end{description}

In addition to the above keys, all of which are required, there are
also some optional keys:

\begin{description}
\item[Mit Unterstützung von] If there has been an assistant that
  helped the supervisor (a lot), name and thank this person here. Do
  not list everyone at the institute.
\item[Weitere Unterstützung] You may (but need not) provide a text (in German)
  that acknowledges a company or an institute that is not part of the
  University of Lübeck where the thesis was mainly written. For
  instance, you could use texts like “\foreignlanguage{german}{Die Arbeit 
  ist im Rahmen einer Tätigkeit bei der Firma Muster GmbH
  entstanden}” or “\foreignlanguage{german}{Die Arbeit ist im Rahmen eines
  Forschungsaufenthalts beim Institut für Dieses und Jenes an der
  Universität Entenhausen entstanden}”. Do not name and thank
  individual persons from the company or foreign institute on the
  title page, instead use the |Acknowledgements| key for that.
\item[Acknowledgements] You can provide a longer paragraph in which
  you thank all the nice people who helped you with the thesis. This
  text will not be shown on the title page, but will be shown rather 
  prominently before the table of contents. It is usually a good idea
  and a nice touch to name everyone who had even a remote positive
  influence on your thesis. 
\end{description}

\section{Including Default and Optional Packages}

Following the |\UzLThesisSetup| command, you can start loading
additional packages that you may need for typesetting your thesis. For
instance, you may wish to load special libraries for drawing finite
automata or you may wish to load a special package for doing fancy
things with tables. Use \LaTeX's |\usepackage| and Ti\emph kZ's
|\usetikzlibrary| commands as usual for this. 

However, the \textsc{uzl-thesis} class loads many packages already and sets
them up in certain ways. This has two effects:

\begin{enumerate}
\item You often do not need to load additional packages at all for
  your thesis since for almost all common problems a suitable package
  will already have been loaded, configured and patched by me.
\item You cannot load a number of packages, because they clash with
  the packages loaded by \textsc{uzl-thesis}. For instance, you cannot use
  the \textsc{pstricks} package since Ti\emph kZ is already loaded,
  you cannot use \textsc{natbib} since \textsc{biblatex} is loaded and
  so on. For all cases where this happens, I had a closer look at the
  alternatives and then explicitly chose the package that is now
  loaded. Even if you like another package better, you will have to
  live with the choice I made.\footnote{If you really want to change
    these choices, finish your studies quickly, finish your PhD
    quickly, become a postdoc, do amazing research, become a professor
    at the University of Lübeck and get voted as Dean of Studies. Then
    you can freely change which packages are preloaded.}
\end{enumerate}

The following packages are preloaded:

\begin{description}
\item[amsmath, amssymb and amsthm] These package are all part of the
  \AmS Math family. In particular, the whole management of
  theorem environments is built on top of these packages, see
  Section~\ref{section-math-theorems} for more details. You can use
  all commands and symbols provided by these packages, see their
  documentation for more information \cite{amsmath}. Any package
  that is incompatible with \AmS Math cannot be used.
\item[babel] This package is loaded and the main and secondary languages
  are setup based on the class options (|german| or |english|). You
  can use the commands provided by \textsc{babel}, see its documentation for
  details \cite{babel}.
\item[biblatex] This package is used for the whole bibliography
  management. There are several alternatives, but this one of the
  places where I made a choice that it not open to
  discussion. Fortunately, \textsc{biblatex} is very powerful and you can use
  its powerful commands in your thesis. See its documentation
  \cite{biblatex}, but 
  also Section~\ref{section-bibliographies}.
\item[fontspec] This package is used for the whole font setup in
  conjunction with Lua\LaTeX\ (and, once more, this is not open to
  discussion). You can use the powerful commands of this package
  \emph{but} you will only rarely need to do so. Mostly, the font
  styling is taken care of via global styling parameters, see
  Section~\vref{section-styling} for more details.
\item[geometry] This package is internally used to setup the page
  layout. \emph{Under no circumstances are you allowed to mess around
    with the margins.} They are perfect the way they are.

  There is a case to be made that there should be an option for
  slightly shifting the pages horizontally depending on whether an odd
  or an even page is printed: When pages printed on both sides
  (double-sided) are bound as a book, it is desirable that the inner
  margin -- where the binding is done -- is larger. However, in my 
  experience, theses are very often read on electronic devices these
  days and, then, it is distracting when the text “jumps around”
  from page to page. For this reason, the default is a fixed layout;
  but you \emph{may} consider doing fancy things for a print version
  that is to be bound as a book. Normally, however, this will not be
  necessary since the margins are large enough that binding is still
  possible without any shifting. 
\item[hyperref] This is a standard package for creating all the
  different links inside the thesis. You can also use its facilities
  for creating links to external material, see the \textsc{hyperref}
  package's manual \cite{hyperref}.

  Note that I took great care to absolutely ensure that there will be
  \emph{no} borders in the \textsc{pdf} file around hyperlinks. The
  default settings of the \textsc{hyperref} package for these borders
  are among the worst and most ugly pieces of typography I have ever
  seen. Whoever made them the default should be ashamed of
  themselves. That means, of course, that \emph{you} should not even
  think of fiddling around with these settings.
\item[listings] This is one of several possible packages for creating
  listings of source code (like the one in
  Listing~\ref{listing-hello-world}). I have chosen this package and
  configured it carefully for your use and strongly recommend that you
  use it (instead of any other package). You will find much more
  information on it in Section~\vref{section-code}.
\item[tikz] This is a powerful package for creating graphics right
  inside the \LaTeX\ document (I happen to be the author, but I do not
  include it out of vanity but because it \emph{is} the standard
  package for creating graphics inside \LaTeX\ documents these
  days). Some Ti\emph kZ libraries are preloaded (|arrows.meta|,
  |positioning| and |shapes|) and a number of styles are predefined
  that you should use, whenever possible, see
  Section~\vref{section-graphics} for more details. 
\item[url] Another standard package that is included for typesetting
  \textsc{url}s more nicely. This is mainly useful for the
  bibliography, where it is used automatically. See its documentation
  for details \cite{url}.
\item[varioref] This is a standard package for creating nicer internal
  cross-references and you can use its facilities freely. See its
  manual \cite{varioref} and also Section~\ref{section-references}.
\end{description}

In addition to the above, the packages \textsc{csquotes},
\textsc{etoolbox} and \textsc{filecontents} are also included, but
only for internal use. It is unlikely that you will need their 
facilities. 



\section{Structuring the Thesis Using Chapters and Sections}

Once you have specified all metadata and loaded all packages that you
need and like, you start with the so called \emph{body} of your
thesis. Normally, this would start with commands like |\maketitle| and
|\tableofcontents| and other commands to create the so called
\emph{front matter,} which is everything before your actual
text. However, with the \textsc{uzl-thesis} class, you do \emph{not}
use any of these commands. Rather, the thesis class will insert all
necessary front matter automatically at the beginning of the thesis --
and it will do so in the right order and with the right page numbering
and so on. This means that you can (and must) directly start the
|{document}| enivornment with the first chapter, which will be the
introduction. Symmetrically to the front matter, the thesis class will
also insert the bibliography automatically at the end. (In particular,
you cannot have a thesis without a bibliography, but that would not be
very scientific anyway.) Once more, please consult
Listing~\vref{listing-hello-world} for an example of how the overall
structure of the thesis must look like. 


\subsection{Chapters and Sections}

The main part of your thesis consists of \emph{chapters}\footnote{In a
  German thesis, use the word “\foreignlanguage{german}{Kapitel}” to refer to a chapter.} (using
the |\chapter| command) and those in turn consist of 
\emph{sections}\footnote{In a  German thesis, use the word “\foreignlanguage{german}{Abschnitt}”
  to refer to a section.} (using the |\section| command). The commands
take the usual arguments as in other document classes.

\newbox\uzlfootnotebox
\setbox\uzlfootnotebox=\hbox{\footnotesize|\setcounter{tocdepth}{2}|}

\subsection{Subsections}

It is also possible to use the |\subsection| command to further
structure sections, but the text will not be shown in the 
table of contents. The reason is that it is generally a bad idea to
nest structures too deeply. Richard Feynman is renown for his 
\emph{Feynman Lectures} in which he introduces the fundamentals of
physics in three thick volumes using only chapters and sections. If
Richard Feynman can explain all of physics using only two levels of
structural nesting, so can you in a hopefully somewhat shorter
thesis. So, while is acceptable to use |\subsection| to structure
sections, do not try to force the titles of the 
subsections into the table of contents and do not try to reference
them.\footnote{Having said this: If it is absolutely necessary that
  you need subsections in the table of contents, try saying
  \box\uzlfootnotebox\ at the beginning of your document.} Also, do not even think of using |\subsubsection|.


\subsection{Lists of Figures and Tables}

In some books you will find lists of all figures and lists of all
tables as separate lists. Frankly, I have never looked at any of them
and consider them perfectly superfluous (and a very unscientific
survey that I conducted shows that all of my colleagues agree). I
guess they are mainly 
included in books (and some theses) because \LaTeX\ makes it easy to
generate them and because they make theses appear longer than they
actually are.

I strongly recommend that you do \emph{not} include such lists in your
thesis. However, in case your advisor \emph{really insists,} you can
say the following (or just one of them, in case you have no figures or
no tables):
\begin{Latex}
\UzLThesisSetup { Abbildungsverzeichnis, Tabellenverzeichnis }
\end{Latex}
This will add the named lists after the table of contents. 


\subsection{Appendices}

It is possible to add appendices using the |\appendix| command
followed by a |\chapter| command. The |\appendix| command will
automatically insert the bibliography first. You will find an example
in the comments of this template, but I will not explain the details
further at this point since I would like to discourage appendices:
In a normal thesis, the main text should explain everything there is
to explain and should also include and relevant data, tables, listings
and proofs. An appendix is usually only needed if there is a lot of
experimental data and the readers is invited to have a closer look at
that data -- but, again, it is then the job of the author to point out
the important features in the main text. I believe it is better not
have an appendix in the thesis and to move all raw data and all source
code into the \emph{electronic supplementary material,} which is
turned in along with the thesis text.

Note, however, that the above arguments only apply to your bachelor's
or master's thesis: Many scientific \emph{papers} actually need an
appendix for various reasons. 


%\chapter{The Thesis Class: Document Text and Document Elements}


\section{Typesetting Basics}

The majority of your thesis will of course consist of “normal text”
like this one. In \LaTeX\ you simply enter this text and the thesis
class does little to interfere with this process. There are only a few
things you should keep in mind and be aware of:

\subsection{Unicode Characters}

The version of \LaTeX\ used by the thesis class, Lua\LaTeX, uses
the Unicode throughout. This means, in particular, that you can and
should enter your main text using the correct and appropriate
Unicode characters. This is true, in particular, for quotation
marks. Supposing you find the characters on your keyboard, you should
write, for instance, in a German text \Latex{Er sagte, »Hallo!«} where
you use the correct Unicode characters directly.\footnote{
  If you are interested, here is a bit of historical context: When
  Donald Knuth first designed \TeX, the Unicode had not 
  even been invented yet. However, Knuth already wanted to typeset for
  instance the German umlaut “Ä” in his texts. His solution was to
  design a special command (\Latex{\\"}) that puts the umlaut on top of
  a character, so “Fräulein” would be typeset as
  \Latex{Fr\\"aulein}. Clearly not an optimal solution for longer
  German texts, but it was a simple solution for individual words --
  and you will still see it a lot in \textsc{bibtex} files.
  
  These days, you should no longer write things like
  \Latex{Fr\\"aulein} in your texts, but just write
  \Latex{Fräulein}. This is not only true for umlauts, but for all
  kind of symbols for which the Unicode offers individual symbols. For
  instance, you can and should also use the (correct) symbols for
  quotation marks -- there are Unicode characters for all of them, see
  also Question~\vref{qu-quotation-marks}.
}

\subsection{Enumerate and Itemize}

You can use the |{enumerate}| and |{itemize}| environments as usual,
but I have changed the vertical spacing compared to the standard
\LaTeX\ styles (nobody understands why these environments insert 
enormous vertical spaces between items in standard \LaTeX). Do not
mess with the spacing.

Note that you can nest the environments, but only up to one level. It
is generally a bad idea to nest the environments and it is definitely
a bad idea to nest them repeatedly. Just do not do that.


\subsection{Line Breaking}

The lines of a paragraph are automatically broken up by \TeX\ and you
usually do not have to worry about line breaking and
hyphenation. I have also changed the way the titles of chapters and
sections are broken both in the main text and in the table of contents
compared to standard \LaTeX\ so that they will usually look
nice. However, there are cases where the line breaks can be improved:
Consider for instance a chapter title like “Review of the System Setup
and the Experiment Design” and suppose that the last word of this
title (“Design”) no 
longer fits on the line. Then \LaTeX\ will do a line break between
“Experiment” and “Design” -- but it would be much better to have the
line break \emph{before} “Experiment” so that the words
“Experiment Design” stay together. You can achieve this by using a
tilde instead of a space between the two words: A tilde is used in
\TeX\ to indicate a “nonbreakable space”, which is exactly what we
want. Thus, you would write:
\begin{Latex}
\chapter{Review of the System Setup and the Experiment~Design}
\end{Latex}
The same is also true for the title of the whole thesis: Use the tilde
in all places of the title where \emph{no} line break should
occur. For instance, the title of the present document was specified
as:
\begin{Latex}
\UzLThesisSetup { Titel auf Deutsch = {
  Vorlage für die \LaTeX-Klasse »uzl-thesis« zur Nutzung bei
  Bachelor-~und Masterarbeiten an der  Universität~zu~Lübeck
} }
\end{Latex}
Here, a nonbreakable space was inserted after “Bachelor-” to ensure
that no line breaking is done exactly there since because of the
hyphen a reader might read “Bachelorund”. Two further nonbreakable
spaces ensure that “Universität zu Lübeck” is always on one line.


\section{Typesetting Mathematics}

\TeX\ is great at typesetting mathematics; indeed, it is nowadays the
gold standard for the correct typesetting of mathematical
text. However, perfectly mastering \TeX's rules for mathematical text
is not easy and this document cannot cover them fully; indeed, it cannot
even give a full introduction and you are kindly requested to read the
\TeX book \cite{Knuth1986}. The following text only highlights the
peculiarities of the thesis class with respect to mathematical text.

First, note that the thesis class preloads the \AmS math package, that
is, the \LaTeX\ packages \textsc{amsmath}, \textsc{amssymb} and
\textsc{amsthm}. This means that you can and should use the commands
offered by these packages, if you need them.


\subsection{Predefined Mathematical Environments}
\label{section-math-theorems}

The thesis class predefines a whole set of environments that you
should use for mathematical statements. A typical way of using them is
as in the following code:
\begin{Latex}
\begin{conjecture}[Goldbach]
  Every even integer $n \ge 4$ is the sum of two primes.
\end{conjecture}
\begin{lemma}
  Every $n \in \{4,6,8,10\}$ is the sum of two primes.
\end{lemma}
\begin{proof}
  We have $4 = 2+2$, $6=3+3$, $8 = 5+3$ and $10 = 5+5$.
\end{proof}
\end{Latex}
This is then typeset as:
\begin{conjecture}[Goldbach]
  Every even integer $n \ge 4$ is the sum of two primes.
\end{conjecture}
\begin{lemma}
  Every $n \in \{4,6,8,10\}$ is the sum of two primes.
\end{lemma}
\begin{proof}
  We have $4 = 2+2$, $6=3+3$, $8 = 5+3$ and $10 = 5+5$.
\end{proof}

As you can see, the numbering of the environments starts anew with
each chapter, but is otherwise consectutive. As someone who has to
read a \emph{lot} of mathemtical texts, I can tell you that this is
the only sensible way of numbering things -- which is why this
numbering style is hardcoded into the thesis class. Do not mess with
it.

The following environments are predefined (in a German thesis, you
\emph{still} use the following environments, they get translated
automatically): 

\begin{description}
\item[theorem] A mathematical statement proved in the
  context of the thesis. Note that even simple statements can be
  theorems.
\item[fact] A mathematical statement that has already been proved in
  a paper. Always cite where it was proved. 
\item[lemma] A mathematical statement that is only needed for
  the proof of another theorem. Just because a statement is not “very
  important” it is not automatically a lemma, but can still be a
  theorem. What makes a statement a lemma is the fact that it is not
  needed outside the context of another theorem. 
\item[corollary] A theorem that follows very easily from the
  previous result.
\item[conjecture] A (mathematical) statement that the author
  suspects to be true, but does not know how to prove.
\item[definition] A (mathematical) definition.
\item[problem] The description of a problem.
\item[example] A single example. Always try to give (good) examples.
\item[examples] Use this for multiple examples in a single environment.
\item[counterexample] For a single counterexample. There is a saying
  that any good definition should be accompanied by at least one
  example and at least one counterexample.
\item[counterexamples] For multiple counterexamples.
\item[observation] A (mathematical) statement that is so
  simple that no proof is given and none should be needed.
\item[remark] Any remark that is so important that needs a number and
  can be referenced. Normal remarks can just be written as part of the
  text. 
\item[proof] Note that this will automatically add an end of
  proof mark at the end.
\end{description}

Additionally, for some of the above there are ``starred versions,''
which are not numbered:

\begin{description}
\item[example*]
\item[examples*]
\item[counterexample*]
\item[counterexamples*]
\item[observation*]
\item[remark*]
\end{description}

Finally, there is the following environment:

\begin{description}
\item[claim*] This is an (unnumbered) mathematical statement
  that is made and proven \emph{inside} the proof of a lemma or
  theorem. 
\end{description}

As two further examples of how these environments are used, here is
another example:

\lstDeleteShortInline|

\begin{Latex}
\begin{problem}\label{problem-vc}
  The problem $p$-\textsc{vertex-cover} is the set of all pairs $(G,k)$,
  where $G = (V,E)$ is a graph, for which there is a set $X
  \subseteq V$ such that $|X| \le k$ and such that $\{u,v\} \in E$
  implies $\{u,v\} \cap X \neq \emptyset$.
\end{problem}

\begin{fact}[\textcite{DowneyF13}]
  The parameterized problem $p$-\textsc{vertex-cover} from
  Problem~\ref{problem-vc} is fixed-parameter tractable.
\end{fact}
\end{Latex}

This is rendered as:

\begin{problem}\label{problem-vc}
  The problem $p$-\textsc{vertex-cover} is the set of all pairs $(G,k)$,
  where $G = (V,E)$ is a graph, for which there is a set $X
  \subseteq V$ such that $|X| \le k$ and such that $\{u,v\} \in E$
  implies $\{u,v\} \cap X \neq \emptyset$.
\end{problem}

\begin{fact}[\textcite{DowneyF13}]
  The parameterized problem $p$-\textsc{vertex-cover} from
  Problem~\ref{problem-vc} is fixed-parameter tractable.
\end{fact}

\lstMakeShortInline[style=code,style=inline,language={[LaTeX]tex},moretexcs={chapter}]|



\subsection{“Theorem” or “Satz”?}

In German, a ``theorem'' is traditionally called ``Satz'' and a
``lemma'' is actually a ``\foreignlanguage{german}{Hilfssatz}''. However, many people feel
that ``Theorem'' sounds more scholary and, thus, use ``Theorem'' and
``Lemma'' also in German texts. In German text, the thesis class uses
``Satz'' and ``\foreignlanguage{german}{Hilfssatz}'' (and
``\foreignlanguage{german}{Folgerung}'' for ``corollary'') by 
default. If you prefer the Greek words, say
\begin{Latex}
\UzLStyle{Greek theorem labels}
\end{Latex}
to change the words to ``\foreignlanguage{german}{Theorem}'',
``\foreignlanguage{german}{Lemma}'' and
``\foreignlanguage{german}{Korollar}'' in 
German texts. 


\subsection{Defining New Theorem Styles}

You can define additional mathematical environments using the
facilities of the \textsc{amsthm} pacakge, which is preloaded and
which is used for the mathematical environments of the
class. Normally, you will not need this. Nevertheless, if you really
do, make sure you define things the same way as in the class file 
(have a look there). For instance, here is how the |{question}| and
|{answer}| environments are defined, which are used a lot in the
second part of this document: 
\begin{Latex}
\theoremstyle{plain}
\newtheorem{question}[theorem]{Question}

\theoremstyle{remark}
\newtheorem*{answer}{Answer}
\end{Latex}

\theoremstyle{plain}
\newtheorem{question}[theorem]{Question}

\theoremstyle{remark}
\newtheorem*{answer}{Answer}

The reason I usually recommend against new environments is that especially
beginners tend to define many new environments whose semantics are
somewhat unclear. I have seen texts in which there are theorems,
lemmas, observations, propositions, facts, corollaries, statements and
more all mixed together. I sincerely doubt that even the author had an
idea what the exact difference between a proposition and a theorem
was supposed to be. Try to stick to the predefined environments, they
suffice. 


\subsection{Display-Style Mathematical Statements}

Mathematical statements can either be given ``inline'' like $a^2+b^2 =
c^2$ here or they can get their own line as in
\begin{align}
  \sin^2 \alpha + \cos^2 \beta = 1
\end{align}

Note that the equations are flushed left if you use standard
commands. Indeed, the only command you should \emph{not} use is
|$$|, see also Question~\vref{qu-math-env} for more details on which
environments to use.

\section{Typesetting (Pseudo) Code}
\label{section-code}

Theses from the area of computer science and related disciplines
sometimes include listings of source code and of pseudo code. The
thesis class preloads the \textsc{listings} package and configures it
for you and you are kindly requested to use it for typesetting both
normal code and pseudo code (that is, algorithms).

Since I reconfigured the \textsc{listings} package quite a bit, let me
point out what you should and should not do when typesetting code:
\begin{itemize}
\item
  Use one of the environments and commands defined by the thesis class
  or, if your favorite language is not defined yet, define a new one
  using the special command |\uzldeflanguageshorthand|.
  For instance, to typeset C++ code, you write 
  \begin{Latex}
\begin{C++}
void main () { std::cout << "Hi!"; }
\end{C++}
  \end{Latex}
  This has the same effect as
  \begin{Latex}
\begin{lstlisting}[language=C++, style=code]
void main () { std::cout << "Hi!"; }
\end{lstlisting}
  \end{Latex}
  In either case, the result is:
\begin{C++}
void main () { std::cout << "Hi!"; }
\end{C++}
  The |style=code| sets up the fonts and the correct highlighting. You
  should always use it, but it is selected automatically by the
  standard environments, see below.
\item
  Environments like |{C++}| take options just like the |{lstlisting}|
  environment. Use these options as explained in the manual of
  the \textsc{listings} package to setup floats, captions, label, line
  numbers, the programming language and so on. However, you should
  \emph{not} use the options to mess with the typesetting nor to
  redefined the fonts or other such evil endeavors. 
\item
  For pseudo code, use the |{Pseudocode}| environment. It is explained
  in more detail \vpageref{label-pseudocode}.
\end{itemize}


\subsection{Language Environments}
\label{label-std-env}

The thesis class defines environments for typesetting a number of
languages. Each of these environments installs appropriate fonts and
select a specific programming language for the syntax
highlighting. Note that all of the environments start with a capital
letter. The available environments are the following:
\begin{itemize}
\item The |{Code}| environment does no special highlighting at
  all. It is useful for ``plain text'' or code the should \emph{not}
  be highlighted:
  \begin{Latex}
\begin{Code}
begin the end of the world // end the begin of the world
\end{Code}
  \end{Latex}
  yields
\begin{Code}
begin the end of the world // end the begin of the world
\end{Code}
\item The |{C}| environment should be used for code in the
  C~programming language. 
  \begin{Latex}
\begin{C}[
  float,
  caption = {
    The first C program from the tutorial in \cite{Kernighan1974}.
  },
  label = {lst-hello}
  ]
main( ) {
        printf("hello, world");
}
\end{C}
  \end{Latex}
  yields Listing~\ref{lst-hello}.
\begin{C}[float, caption = {The first C program from the tutorial in \cite{Kernighan1974}.},  label = {lst-hello}]
main( ) {
        printf("hello, world");
}
\end{C}
\item The |{C++}| or |{Cpp}| environments should be used for C++
  code. They have the same effect.
\item Use |{Java}| for Java code.
\item Use |{Latex}| for \LaTeX\ code (note the spelling of the word
  |Latex|: A capital L, but otherwise lowercase letters. The command
  |\LaTeX| yields, in contrast, the word “\LaTeX”).
\item Use |{Matlab}| for \textsc{matlab} code.
\item Use |{Pseudocode}| for pseudo code and algorithms, see below for more details.
\item Use |{Sql}| for \textsc{sql} code (note the spelling once more:
  A capital S, but otherwise lowercase letters).
\end{itemize}

The \textsc{listings} package supports a large number of additional
languages, but I have decided to pre-define only the above. You can
easily define further languages using the following command:

\begin{Latex}[style=escapemath]
\uzldeflanguageshorthand{$\langle\mathit{Environment}\rangle$}{style=code,language=$\langle\mathit{language}\rangle$}
\end{Latex}

Here, $\langle\mathit{Environment}\rangle$ should of course be the
name of the environment (preferably with a leading capital letter) and
$\langle\mathit{language}\rangle$ must be a language name that the
\textsc{listings} package understands. As an example, here is a
definition for Pascal:

\begin{Latex}
\uzldeflanguageshorthand{Pascal}{style=code,language=pascal}
% Now you can say
\begin{Pascal}
sum := 0;
for i := 1 to 100 do
begin
  sum := sum + i
end
\end{Pascal}
\end{Latex}
and get
\uzldeflanguageshorthand{Pascal}{style=code,language=pascal}
% Now you can say
\begin{Pascal}
sum := 0;
for i := 1 to 100 do
begin
  sum := sum + i
end
\end{Pascal}


\subsection{Pseudocode Environment}
\label{label-pseudocode}


The |{Pseudocode}| environment is special compared to the other
languages: A different font is used and the spacing is different. You
use it as follows:

\begin{itemize}
\item You write down your pseudo code and algorithms in a “literal”
  way, which means that you write them down inside the environment
  more or less the way they should look like later. Here is a simple
  example:
\begin{Latex}
\begin{Pseudocode}
print ``Please enter a number below 10.''
input $n$
if $n > 9$ then
  print ``Too high!''
else
  print ``Thank you!''
\end{Pseudocode}
\end{Latex}
  The output is then:
\begin{Pseudocode}
print ``Please enter a number below 10.''
input $n$
if $n > 9$ then
  print ``Too high!''
else
  print ``Thank you!''
\end{Pseudocode}
\item
  As can be seen in the example, you can and should put variables and
  mathematical expressions inside the \LaTeX\ math environment using
  |$|'s. %$
  Indeed, the |{Pseudocode}| environment allows you to “escape” to
  \LaTeX\ math mode at any point. I recommend writing 
  “|$n > 9$|” rather than “|$n$ > $9$|” or just “|$n$ > 9|” or
  “|n > 9|” since only the first version will get all the spacing and
  fonts correct. See also Question~\ref{qu-math-font}.
\item
  Use |``| and |''| to start and end strings.
\item
  Use |//| to start comments. There are no multi-line comments
  in pseudo code. The comments will not flush right. If you really,
  really want this, you can do tricky things with escapes and
  |\hfill|. I will not tell you how to do this.
\item
  Some common keywords are setup and highlighted automatically. See
  Listing~\vref{listing-merge} for an example of how to add or remove
  words from the keyword list.
\item
  The character sequence “|<-|” is redefined so that it yields a
  left-pointing arrow (“$\gets$”). This is usual way to typeset
  assignments in pseudocode: Say
  \begin{Latex}
\begin{Pseudocode}
algorithm compute_sum ($n$)
   // This function computes the sum of the first $n$ numbers.
   if $n<0$ then
      throw ``negative number''
   $s$ <- $0$
   foreach $i \in \{1,\dots,n\}$ do 
      $s$ <- $s + i$
   return $s$
\end{Pseudocode}
\end{Latex}
  to get
\begin{Pseudocode}
algorithm compute_sum ($n$)
   // This function computes the sum of the first $n$ numbers.
   if $n<0$ then
      throw ``negative number''
   $s$ <- $0$
   foreach $i \in \{1,\dots,n\}$ do 
      $s$ <- $s + i$
   return $s$
\end{Pseudocode}
\end{itemize}

\begin{Pseudocode}[caption={A merge implementation in pseudocode. For
    this example, two additional keywords were added using the option
    \Latex{morekeywords = \{append, remove\}} while one keyword got
    deleted using \Latex{deletekeywords = to}. This causes the words
    ``append'' and ``remove'' to be highlighted, while ``to'' is not
    highlighted.},
  float, label=listing-merge, morekeywords={append,remove}, deletekeywords={to}]
procedure merge($A$, $B$)
  $C$ <- empty list
  while $A$ not empty and $B$ not empty do
  if $A.\mathit{first} < B.\mathit{first}$ then
      append $A.\mathit{first}$ to $C$
      remove first from $A$
    else
      append $B.\mathit{first}$ to $C$
      remove first from $B$

  append remaining elements from $A$ and $B$ to $C$
  return $C$    
\end{Pseudocode}



\subsection{Floating Code}


In \TeX, “floating“ text is a figure, table or listing that is not
necessarily inserted at the point where it is in \LaTeX\ source code,
but at a possibly later point so that it fits nicely on the page. For
listings, you can use the |float| key to turn the listing into a
floating listing. For instance, the code
\begin{Latex}
\begin{C++}[caption={A ...}, label={lst-1}, float, numbers=left]
#include <iostream>

void main()  // Where did the arguments go?
{
  std::cout << "Hello World!" << std::endl;
}
\end{C++}
\end{Latex}
%
yields Listing~\vref{lst-1} which is right behind this paragraph in
the \LaTeX\ manuscript, but which will not necessarily be right behind 
this paragraph in the \textsc{pdf} file.

\begin{C++}[caption={A “floating” hello world program in C++. It was
    typeset using the C++ environment with the key \Latex{caption}
    set to the text you are reading now, the key \Latex{float} set to
    make the listing “float” and the key \Latex{numbers = left} to add
    some line numbers.}, label={lst-1}, float, floatplacement=t, numbers=left]
#include <iostream>

void main()  // Where did the arguments go?
{
  std::cout << "Hello World!" << std::endl;
}
\end{C++}

Note that you should always use the |caption| key to provide an
explaining caption for floating listings. You can also use the key for
non-floating listings, but I advise against that: better explain
things in the main text directly before or after the listing.


\subsection{Inline Listings}

It is possible to “inline” listings, which just means that code is
displayed in the middle of a normal paragraph and not in a separate
block. You use the commands of the same name as the corresponding
language environments for this: For instance,
while the environment |{Java}| is used to typeset a block of Java
code, 
the command |\Java{int i;}| will typeset the code right inside the
text, namely like this: \Java{int i;}. Similarly, |\C{int i;}| will
yield \C{int i;} while |\Code{int i;}| will yield \Code{int i;} and
|\Pseudocode{int i;}| will yield \Pseudocode{int i;}.

Internally, all of these command just call |\lstinline|. For instance,
|\Java|$\langle\mathit{text}\rangle$ is mainly a shorthand for
\begin{Latex}[style=escapemath]
\lstinline[style=code, language=java, style=inline]$\langle\mathit{text}\rangle$
\end{Latex}
Any options passed to command like |\Java| will be passed on to
|\lstinline|. Note that the |style=inline| is used to setup different
fonts for inline text. Do not mess with them. Finally, note that
instead of curly braces you can also use a so called delimiting
character, see the documentation of the \textsc{listings} package for
more details.


\subsection{Miscellaneous}

There is a special style called |escapemath| that allows you to escape
to math mode inside a listing using |$|. It is switched on by default for %$
pseudo code, but not for normal code. It is useful for inserting
meta-variables into a listing:
\begin{Latex}
\begin{C++}[style=escapemath]
void $\langle \mathit{identifier}\rangle$ (int i)
{
  return i == 0 ? 1 : 1 + $\langle \mathit{identifier}\rangle$(i-1);
}
\end{C++}    
\end{Latex}
yields
\begin{C++}[style=escapemath]
void $\langle \mathit{identifier}\rangle$ (int i)
{
  return i == 0 ? 1 : 1 + $\langle \mathit{identifier}\rangle$(i-1);
}
\end{C++}    

In inline code, spaces are shown by default. You can switch this off
using the key |showspaces=false| as in |\C[showspaces=false]{int i;}|,
which yields \C[showspaces=false]{int i;}.


\section{Creating Figures and Tables}

With the thesis class, you add figures and tables using the standard
\texttt{figure} and \texttt{table} environments. You should always add
a caption to a figure and the caption should be below it, while the
caption of a table should be above it. You should always label the
figure and you \emph{must} reference all figures at least once in the
text. See Question~\vref{qu-captions} for some hints on what to write
in captions. 


\subsection{Figures}

The following code is typical code for creating a figure. See
Figure~\vref{fig-ode1} for the resulting typeset figure. Note that the
font size is set to |\small| inside figure (and, also, inside tables).

\begin{Latex}
\begin{figure}[htbp]
  \centering
  \textcolor{black!10}{\vrule width3cm height 3cm}
  \caption{Ode to Malewitsch.}
  \label{fig-ode1}
\end{figure}
\end{Latex}

\begin{figure}[htbp]
  \centering
  \textcolor{black!10}{\vrule width3cm height 3cm}
  \caption{Ode to Malewitsch.}
  \label{fig-ode1}

\end{figure}


\subsection{Tables}

For tables, I recommend the following:
\begin{itemize}
\item Add horizontal lines around the table using the
  special command |\uzlhline|, see the example below, and possibly
  after the first line of the table. The exact size
  and color of the line will depend on the design, but you must insert
  it “by hand”. Adding these lines will make tables visually
  consistent with listings and the rest of the thesis. However, you
  may also leave out the lines.
\item
  Do not add any more or other lines. There is a saying in typography:
  “There is not a single table that cannot be improved by removing all
  lines.” Lines are distracting and good spacing is a much better way
  of creating a visually pleasing table.
\item
  Use the command |\uzlemph| with the text of column headers and (if
  needed) row headers. This command, which can also be used elsewhere,
  will typeset its argument in the same way as headlines are typeset
  (but in the usual font size). This will cause table heads to be
  visually consistent with the rest of the thesis design.
\end{itemize}

As an example, here is the code for Table~\vref{fig-tab1}:
\begin{Latex}
\begin{table}[htpb]
  \caption{Sounds made by different kinds of animals...}
  \label{fig-tab1}
  \centering
  \begin{tabular}{lp{5cm}}
    \uzlhline
    \uzlemph{Animal} & \uzlemph{Sound} \\ \uzlhline
    Cat & Meow \\
    Dog & Wuff or bark\\ \uzlhline
  \end{tabular}
\end{table}
\end{Latex}

\begin{table}[htpb]
  \caption{Sounds made by different kinds of animals, based on the
    reports by children of the age range 5 to~10. While the number~$n$
    of children asked to report was small ($n=1$) and the results are,
    thus, not fully representative of the general population, the
    author would like to point out that the ethics committee approved
    this survey.}
  \label{fig-tab1}
  \centering
  \begin{tabular}{lp{5cm}}
    \uzlhline
    \uzlemph{Animal} & \uzlemph{Sound} \\ \uzlhline
    Cat & Meow \\
    Dog & Wuff or bark\\ \uzlhline
  \end{tabular}
\end{table}


\section{Creating Graphics}
\label{section-graphics}

\subsection{External Graphics}
\label{section-external}

Graphics (like plots, images, drawings or other data visualizations)
can be added to a \LaTeX\ document in two ways: First, you can include
an \emph{external graphic} like a \textsc{pdf} file or a \textsc{jpg} 
file. Second, you can use  “describe your graphic using \LaTeX\
commands”. We discuss external graphics next, internal ones later on.

You include an external graphics using the |\includegraphics| command,
which is a standard \LaTeX\ command. There is no need to include any
packages for this, it is available automatically. For instance, you
could say:

\begin{Latex}
The university slogan \includegraphics{uzl-thesis-logo-slogan.pdf} in a sentence.
\end{Latex}
to get: “The university slogan \includegraphics{uzl-thesis-logo-slogan.pdf} in a
sentence.” 

As can be seen, the effect of the |\includegraphics| command is to
directly include the graphic at the very position in the line where
the command is used. Indeed, from \TeX's point of view, an external
graphic is indistinguishable from a black rectangle of the same size
as the graphic.

You will rarely wish to put a graphic in the middle of a sentence
(although there are applications). Instead, you will usually place it
inside a |{figure}| environment: Recall that it is the job of the
environment to creating “floating” text with a caption -- and it is
then the job of |\includegraphics| to replace the “text” by a
picture. Here is an example of how you will usually do this:

\begin{Latex}
\begin{figure}[htpb]
  \centering
  \includegraphics{uzl-thesis-logo-uzl.pdf}
  \caption{The logo of the University of Lübeck. It consists...}
  \label{fig-logo}
\end{figure}
\end{Latex}
The result is Figure~\vref{fig-logo}.
\begin{figure}[htpb]
  \centering
  \includegraphics{uzl-thesis-logo-uzl.pdf}
  \caption{The logo of the University of Lübeck. It consists of the
    university's seal together with the text “Universität zu
    Lübeck”. The corporate design manual of the university requires
    this logo to be put at the upper left corner of title pages of
    university publications.}
  \label{fig-logo}
\end{figure}

The |\includegraphics| command takes many options, the most important
of which are likely |height| and |width|. These allow you to scale the
graphic to a given height or width. \emph{Avoid these options whenever
  possible.} The reason is that most graphics have a natural size
(such as the logo) in which the text and fonts in the graphic are at
the correct sizes. Any scaling will cause the graphic to become too
large or too small. \emph{Scaling is evil} and you will find more
comments on this in Question~\vref{qu-scaling}. All professors I
know find scaled-down graphics with unintelligible text \emph{among
  the most irritating things a student could  possibly do.}  

This means that when you \emph{create} graphics with another program,
make \emph{sure} that any text in the external graphic has the same
size as normal text in the thesis and that \emph{no} scaling is needed.

\subsection{Inline Graphics via Ti\emph kZ}
\label{section-tikz}

The alternative to external graphics are \emph{internal}
graphics. They are created using special \LaTeX\ commands such as the
following:
\begin{Latex}
\tikz \draw [->] (0,0) -- (1cm, 2mm);
\end{Latex}
which yields \tikz \draw [->] (0,0) -- (1cm, 2mm); when used in a
paragraph. A more complex example would be 
\begin{Latex}
\tikz [baseline, anchor=base] {
  \node [block = emph]                             (h) {Hello};
  \node [small node = emph blue, right = 5mm of h] (w) {Welt};
  \draw (h) edge[bend left=15, <->] (w);
}
\end{Latex}
which yields
\tikz [baseline, anchor=base] {
  \node [block = emph]                              (h) {Hallo};
  \node [small node  = emph blue, right = 5mm of h] (w) {Welt};
  \draw (h) edge[bend left=15, <->] (w);
}.

The Ti\emph kZ package is used for these inline graphics and it is
loaded automatically -- so \emph{if} you are going to create inline
graphics, use Ti\emph kZ. If you wish to learn Ti\emph kZ, please read
the tutorials from the manual \cite[Part I]{Tantau2019}.

The thesis class sets up some styles for Ti\emph kZ that you can
either explicitly use or that are generally set up. For instance, the
default arrow tip is setup according to the chosen design as well as
the standard line width. You usually do not need to worry about these
automatic settings.



\subsection{Predefined Ti\emph kZ Styles}

There are several styles that are predefined and that you are
“invited” to use: 

\begin{description}
\item[base shape] This is a style on which the styles described next
  are based. It is to be used with the |\node| command and will, for
  instance, fill the node with white color and will draw a thick
  border around it. The color that is used for the border can be
  passed as an argument, but see below for which colors you should
  use. Here is a simple example how this style is used:
  \begin{Latex}
\tikz \node [base shape, ellipse] {Hello};    
  \end{Latex}
  yields
\tikz [baseline, anchor=base] \node [base shape, ellipse] {Hello};.

  The default color used for shapes is defined by the design (either
  black or the color |Ozeangruen|, which is the university's corporate
  design color, depending on the design). 
\item[small shape] This style can be used in addition to |base shape|
  and will change the font to a smaller size and will reduce 
  the inner seperation:
  \begin{Latex}
\tikz \node [base shape, small shape, ellipse] {Hello};    
  \end{Latex}
  yields
\tikz [baseline, anchor=base] \node [base shape, small shape, ellipse] {Hello};.
\item[block] A rectangular node: |\tikz \node [block] {Hello};| yields
\tikz [baseline, anchor=base] \node [block] {Hello};.
\item[small block] A smaller version: Saying
|\tikz \node [small block] {Hello};|
  will now give
\tikz [baseline, anchor=base] \node [small block] {Hello};.
\item[node] A circular node, especially in a graph: |\tikz \node [node] {$n$};|
  yields
\tikz [baseline, anchor=base] \node [node] {$n$};.
\item[small node] Small version: |\tikz \node [small node] {$n$};|
  yields
\tikz [baseline, anchor=base] \node [small node] {$n$};.
\item[tiny node] A small, unnamed circular node: |\tikz \node [tiny node] {};|
  yields
  \tikz \node [tiny node] {};.
\item[paper] This is a special style that installs a “paper-like
  background” as fill color. I recommend using this style as a
  background for all cases where you wish to create the impression of
  showing an “original text” from another paper, that is, where you
  wish to show that the text or graphic “looks like this in the
  original”. Consider for instance,
  \begin{Latex}
\begin{figure}[htpb]
  \centering
  \tikz \node [paper] {\includegraphics{uzl-thesis-logo-uzl.pdf}};    
  \caption{The logo of the University of Lübeck, but now...}
  \label{fig-logo-paper}
\end{figure}
\end{Latex}
which yields Figure~\ref{fig-logo-paper}.
\begin{figure}[htpb]
  \centering
  \tikz \node [paper] {\includegraphics{uzl-thesis-logo-uzl.pdf}};    
  \caption{The logo of the University of Lübeck, but now with a
    “paper-like background”. Note how, compared to
    Figure~\ref{fig-logo}, a much stronger impression is created that
    this figure depicts something printed.}
  \label{fig-logo-paper}
\end{figure}
\end{description}

An example of how many of these styles can be used is shown in
Figure~\vref{fig-tikz-white}. 

\newcommand\examplepicture{

  \tikzset{anchor=mid,yscale=.85}

  %
  % Merge sort
  %

  \node at (0,5.5) {\uzlemph{Merge sort visualization}};
  
  \node [block] (input) [minimum width=6cm] at (0,4.25)        { $A_1,\dots,A_n$ };
  \node [block] (l1)    [minimum width=2.8cm] at (-1.6,3)      { $A_1,\dots,A_{n/2}$ }; 
  \node [block] (l2)    [minimum width=2.8cm] at (1.6,3)       { $A_{n/2+1},\dots,A_n$ }; 

  \node (l11) at (-2.2,1.75) {$\vdots$};
  \node (l12) at (-1,1.75) {$\vdots$};
  \node (l21) at (1,1.75) {$\vdots$};
  \node (l22) at (2.2,1.75) {$\vdots$};

  \node [small node] (m1)  at (-1.6,1) {$\mu$};
  \node [small node] (m2)  at (1.6,1) {$\mu$};
  
  \node [block=emph] (b1)    [minimum width=2.8cm] at (-1.6,0) { $B_1,\dots,B_{n/2}$ }; 
  \node [block=emph] (b2)    [minimum width=2.8cm] at (1.6,0)  { $B_{n/2+1},\dots,B_n$ }; 

  \node [small node=emph, label=right:(merge)] (m3)  at (0,-1) {$\mu$};
  \node [block] (output) [minimum width=6cm] at (0,-2)         { $B_1,\dots,B_n$ };
  
  \draw (input)  edge [->] (l1)
                 edge [->] (l2)
        (l1)     edge [->] (l11)
                 edge [->] (l12)
        (l2)     edge [->] (l21)
                 edge [->] (l22)
        (m1)     edge [<-] (l11)
                 edge [<-] (l12)
        (m2)     edge [<-] (l21)
                 edge [<-] (l22)
        (m1)     edge [->] (b1)
        (m2)     edge [->] (b2)
        (m3)     edge [<-] (b1)
                 edge [<-] (b2)
                 edge [->] (output);

  \draw [overlay,very thick, structure] (3.5,-3.5) -- (3.5,5);

  % 
  % The blocks
  %

  \node at (7,5.5) {\uzlemph{Predefined styles}};
  
  \foreach \style [count=\i] in {
    block,
    block=emph,
    block=emph blue,
    block=emph green,
    block=emph black,
    small block,
    small block=emph,
    small block=emph green,
  } {
    \expandafter\node\expandafter [\style, anchor=mid west] at (4,5.25-\i) {\strut\ttfamily\style};
  }
  
  \foreach \style/\what [count=\i] in {
    node/$n$,
    node=emph/$n$,
    node=emph blue/$n$,
    node=emph green/$n$,
    node=emph black/$n$,
    small node/$n$,
    small node=emph/$n$,
    tiny node/,
  } {
    \expandafter\node\expandafter [\style, anchor=mid, label=right:\ttfamily\style] at (8.25,5.25-\i) {\what};
  }  
}


\begin{figure}[htpb]
  \centering
  \tikz[thesis outline shapes]{\examplepicture}
  \caption{An example visualization created with Ti\emph kZ, see the
    template source for the code details. In the graphic, a number of
    predefined styles are used (like \Latex{node} or \Latex{small block}), each of which can be passed an optional color. These
    styles are setup automatically to produce visually pleasing shapes
    that go well with the overall layout and fonts.}
  \label{fig-tikz-white}
\end{figure}



\subsection{Predefined Colors}

The thesis class defines a number of colors that you should use in
graphics. You should \emph{not} use colors like |red| or |green|: Pure
green is a very light color and text in this color is hard to read on
paper and impossible to read in an electronic document. Instead of
pure green, a rather dark version of green must be used. In contrast,
pure blue is already rather dark and only needs to be darkened very
slightly. The following colors have been setup to provide a uniform
contrast against a white background:

\begin{description}
\item[emph]
  A red color.
  Used in an outline
  \tikz[baseline] \draw [thick,emph] (0,-.2ex) rectangle ++(3.2ex,1.6ex);
  and filled
  \tikz[baseline] \fill[emph] (0,.5ex) circle[radius=.8ex];
\item[emph red] This is the same as |emph|.
\item[emph green]
  Used in an outline
  \tikz[baseline] \draw [thick,emph green] (0,-.2ex) rectangle ++(3.2ex,1.6ex);
  and filled
  \tikz[baseline] \fill[emph green] (0,.5ex) circle[radius=.8ex];
\item[emph blue]
  Used in an outline
  \tikz[baseline] \draw [thick,emph blue] (0,-.2ex) rectangle ++(3.2ex,1.6ex);
  and filled
  \tikz[baseline] \fill[emph blue] (0,.5ex) circle[radius=.8ex];
\item[emph black] This is just black:
  Used in an outline
  \tikz[baseline] \draw [thick,emph black] (0,-.2ex) rectangle ++(3.2ex,1.6ex);
  and filled
  \tikz[baseline] \fill[emph black] (0,.5ex) circle[radius=.8ex];
\end{description}

Looking for more colors? Think carefully whether you really need more:
It is hard to remember too many colors as a reader. It may be better
to use a different style (like thicker lines) for
purposes of further differentiation.



\subsection{Designs for Ti\emph kZ Graphics}

In addition to the designs for the whole thesis, see
Section~\ref{section-styling}, there are also three designs for
graphics. Each of them redefines |base shape|, resulting in a
different “look”:


\begin{description}
\item[thesis outline shapes] This Ti\emph kZ style defines
  |base shape| and all styles built on top of it (like |block| and |node|)
  to a white background and simple thick line around the shape. For
  instance:
\begin{Latex}
\tikzset{thesis outline shapes} % generally set the design
\tikz \node [block] {Hello};    
\end{Latex}
yields
{%
\tikzset{thesis outline shapes}% generally set the design
\tikz[baseline, anchor=base] \node [block] {Hello};%    
}. See Figure~\vref{fig-tikz-white} for a larger example.
\item[thesis box shapes] This style defines |base shape| similarly to
  the outline style, but fills the shapes with a light background:
\begin{Latex}
\tikzset{thesis box shapes} % generally set the design
\tikz \node [block] {Hello};    
\end{Latex}
yields
{%
\tikzset{thesis box shapes}% generally set the design
\tikz[baseline, anchor=base] \node [block] {Hello};%    
}. See Figure~\vref{fig-tikz-box} for an example.
\item[thesis flat shapes] This style redefines |base shape|
  differently: The shapes are only filled and no border is 
  drawn. This creates a stylish “flat” look:
\begin{Latex}
\tikzset{thesis flat shapes} % generally set the design
\tikz \node [block] {Hello};    
\end{Latex}
yields
{%
\tikzset{thesis flat shapes}% generally set the design
\tikz[baseline, anchor=base] \node [block] {Hello};%   
}. See Figure~\vref{fig-tikz-flat} for a larger example.
\end{description}




\begin{figure}[htpb]
  \centering
  \tikz[thesis box shapes]{\examplepicture}
  \caption{The same visualization as in Figure~\ref{fig-tikz-white},
    but with the \Latex{thesis box shapes} option set. As can be
    seen, this option causes the predefined shapes to be filled. It is
    a matter of taste whether one prefers this over the \Latex{outline}
    style from Figure~\ref{fig-tikz-white}.}
  \label{fig-tikz-box}
\end{figure}

\begin{figure}[htpb]
  \centering
  \tikz[thesis flat shapes]{\examplepicture}
  \caption{Once more, the same visualization as in Figure~\ref{fig-tikz-white},
    but now with \Latex{thesis flat shapes} option set. This option
    creates more ``flat'' shapes (without a border). Once more it is a
    matter of taste what one prefers.}
  \label{fig-tikz-flat}
\end{figure}



\section{Creating Cross References, Citations and the~Bibliography}

\label{section-references}
\label{section-bibliographies}

“We all stand on the should of giants” is a saying that refers to the
fact that, in science, you never start from scratch: Your work always
builds on that of many, many people before you. It is thus not only
permissible, but actually required that you refer to the work of other 
people using \emph{citations} -- how this is done with the thesis
class is explained below. In addition, apart from citing the work
of other people, you will also often wish to point to different parts
of your own thesis. The method of creating such \emph{cross
  references} is also explained in this section.


\subsection{Citations}

In your thesis, you will and should use citations. This is done in
\LaTeX\ using the \Latex{\cite} command: If I wish to say that the
\TeX\ typesetting system was written by Donald Knuth, I could write
\begin{Latex}
The \TeX\ system~\cite{Knuth1986} is due to Donald Knuth.
\end{Latex}
and would get “The \TeX\ system~\cite{Knuth1986} is due to Donald
Knuth.” The |\cite| command looks up the \emph{bibliography key}
|Knuth1986| in the list of bibliography entries (see below) and then
inserts the citation into the text and the bibliography entry into the
bibliography. All of this happens automatically, you just need to
specify the entry in the list of bibliography entries and you need to
add the citations at all points where you cite a book or an article in
the main text.

The thesis class uses the \textsc{biblatex} package for the
bibliography handling \cite{biblatex}. You can use the different
commands offered by the package, such as for instance |\citeauthor| to
just cite the author names. There are a lot of commands, it is a powerful
package.   


\subsection{Bibliography Entries (Bib\TeX\ Entries or Records)}

In order to cite a book or an article, you say |\cite{|$\langle
  \mathit{key}\rangle$|}| in the text, where $\langle
\mathit{key}\rangle$ must be an entry in the list of bibliography
entries. For historic reasons, these entries come in a weird format
and must be processed using a special program called Bib\TeX.

I have tried to simplify the whole Bib\TeX\ business as much as
possible for you:
\begin{itemize}
\item With the thesis class, you can place the bibliography entries
  right inside your (\LaTeX) manuscript: At any point in the
  manuscript (I recommend somewhere near the end), you can use the
  |{bibtex entries}| environment. Inside this environment, you place
  all the bibliography entries using the \textsc{bibtex} format (if
  you are unsure about this format, look up the documentation of
  \textsc{biblatex}, but the format is fairly
  self-ex\-pla\-na\-to\-ry). Listing~\ref{listing-bibtex} shows an example of
  how this works.
  \begin{Latex}[
    float,
    caption = {
      Typical use of the \Latex{\{bibtex entries\}} environment inside a
      thesis manu\-script.
    },
    label = listing-bibtex]
\begin{document}
...
The \TeX\ system~\cite{Knuth1986} is due to Donald Knuth.
...
\begin{bibtex entries}
@Book{Knuth1986,
  author = 	 {Donald Erwin Knuth},
  title = 	 {The \TeX book},
  publisher = 	 {Addison-Wesley},
  year = 	 {1986},
}
...
\end{bibtex entries}
\end{document}
  \end{Latex}
\item You will then have to run |lualatex| on your manuscript and then
  run |bibtex| on the manuscript and then run |lualatex| once
  more. That should do the trick. (The effect of the
  |{bibtex entries}| environment is that when \LaTeX\ 
  is run on your manuscript, the text inside the environment gets
  written into a special file that is then processed by
  |bibtex| later on.)
\item Actually, the “right” program to use with \textsc{biblatex} is
  not |bibtex|, but |biber|. However, the advantages of |biber| are
  only slight and setting it up correctly can be nontrivial. If you do
  wish to use |biber|, you must say 
  \begin{Latex}
\UzLThesisSetup{biblatex={backend=biber}}
  \end{Latex}
  Do this only if you know what you are doing.
\item
  If, for whatever reasons, you do not wish to use the
  |{bibtex entries}| environment, you can and must use the
  |\addbibresource| command of the \textsc{biblatex} package. Do this
  only if you know what you are doing.
\end{itemize}


\subsection{The Bibliography}

The bibliography itself (the “list of references”) is inserted
automatically at the end of the thesis. Do not mess with this.

The bibliography style (the way the bibliography “looks”) is also not
up to you, it is dictated by university regulations. You \emph{must} 
use the citation style of the journal \emph{Cell} for all theses
written in the area of the natural sciences (like Molecular Life
Science) and you \emph{can} also use so-called “numbered” citations  
for the all other subjects (but you can also use the citation style of
\emph{Cell} for them). The bottom line is that you must say in the
preamble 
\begin{Latex}
\UzLThesisSetup{Alphabetische Bibliographie}
\end{Latex}
for a thesis in the area of the natural sciences and you can,
alternatively, say
\begin{Latex}
\UzLThesisSetup{Numerische Bibliographie}
\end{Latex}
for all other subjects. These commands will install the correct
citation modes. Do not mess with them.\footnote{It really took me a
  \emph{long} time to get them right.}



\subsection{Cross References}

In scientific texts, you will often wish to reference an element of
the text itself at some other place: For instance, in the text you
wish to write something like “As Figure~3.2 shows, there is no
evidence for\dots“ where “Figure~3.2” is, of course, a figure in your
thesis. In \LaTeX, you use the commands |\label| and |\ref| for these
purposes, see the \LaTeX\ manual \cite{Lamport1994} for more details,
and the \textsc{varioref} package is also automatically loaded. In
particular, you can use the useful |\vref| command, see
\cite{varioref} for the documentation. 



\chapter[Questions~\&~Answers:
  How~to~Write~a~Thesis~That~Is~a~Pleasure~to~Read] 
  {Questions~\&~Answers: How~to~Write a Thesis That~Is a
  Pleasure~to~Read} 
\label{chapter-tips}

Suppose that, like me (Professor Tantau), you are asked to grade a
thesis that a hopeful student just turned in. Suppose that, like me,
you have to grade a \emph{lot} of theses (at least one per month) and
you start reading 
it. Certainly, you strive to be as fair and even-minded as humanly
possible when it comes to grading the thesis. In particular, you have
the firm intention of determining your final grade based on the
quality of the scientific contribution that the thesis makes.

As you start reading, you stumble over a number things. It starts with
the table of contents that seems a bit confusing and after having read
both the abstract and the one-page introduction you are still not
quite sure what the author actually achieved and start to wonder
“What is the point of all of this? And what did the guy actually
\emph{do} during the six months?” But you read on. 

As you continue reading, your reading flow gets interrupted repeatedly
by spelling mistakes, each of which 
makes you cringe a little. At first you think that
the student might have introduced them during some
last-minute revision work, but then you stop reading and start marking 
the spelling and grammar mistakes on five consecutive pages. You are 
somewhat disappointed when you find that there are about ten mistakes
per page and the number does not appear to drop significantly as the
text continues. By multiplying the number of mistakes per page with
the number of pages, you extrapolate that the total number of mistakes
must be in the range of 200 to 400 spelling mistakes. This makes you
pretty sad and even a bit angry since you start to wonder “Why do I
have to read this? Why couldn't the author be bothered to run a
spellchecker on this text during the \emph{six months} of working on
it?” But you read on.

As you have a closer look at the main part of the text, you get
determined to find out what author's main point actually is. It is a
struggle since the author sometimes explains trivial stuff over two
pages but then does not provide a single example with the main
definition. Nor with any of the later results. But there are some
tables, although it is a bit unclear what they actually show. Anyway,
you are a professor and a professor can do anything, so you can also
understand this text. It turns out that the trick is to quickly read
the one paper that the student cites a zillion times -- probably the
only paper the student actually read. With the knowledge of that
paper, you realize that the thesis actually contains some nice ideas:
perhaps not terribly deep results, but definitely nice ideas.

Here is the problem you face: How to grade a thesis that is
literally littered with spelling mistakes, that is a mess
concerning structure and exposition, that has a terrible
bibliography, that forces you to read the original literature in order
to understand what it is all about -- and that contains really nice
scientific ideas?

Welcome to my world. 
“Nice scientific ideas” can be graded in almost any way one likes: In
a review, one can write that “the thesis certainly contains nice
ideas, but utterly fails to demonstrate their actual relevance” or one
can write “the thesis certainly contains nice ideas and these are
likely to open new and interesting future approaches -- a full
demonstration of the practical relevance of the ideas is of course
beyond the scope of a thesis”.  My point is that you, dear students,
\emph{do not want to infuriate 
  the person who grades your thesis!} You want to turn in a thesis
that is \emph{a pleasure to read!} This will determine which of the
two reviews will be written.

Of course, if your thesis proves $\mathrm P \neq \mathrm{NP}$ or
presents a cure for cancer, you can turn in total gibberish, you can
use a spelling that would make even your friends cringe when used in a
chat, you can write your thesis on a napkin and the thesis can 
have two pages or 1000 pages -- you will still get an outstanding
grade. The other way round, no matter how stylish your thesis looks,
how carefully it is phrased and how nicely you have wrapped the
binding with leather -- if you do not present solid scientific work in
your thesis, you will still fail.

But in all other cases, you will just need to make an effort to
create a well-written thesis. In the rest of this chapter, I try to
give some hints on how to do this and will try to cover most of the famous
\emph{frequently asked questions} that I am, indeed, frequently
asked. As with any recommendations, you can ignore them, but you do so
at your own risk. But above all, remember that you should \emph{always
  listen to your adviser!} 


\section{Q\&A: Thesis Length}

\begin{question}
  How many pages should my thesis have?
\end{question}

\begin{answer}
  There is no fixed length for a thesis, but for a
  bachelor's thesis the rule of thumb is 20 to 40 pages with 30 being
  ideal. For a master's thesis, aim for twice that, so 40 to 80 pages
  with 60 being ideal. The pages include everything following the
  table of contents (the pages numbered in Arabic in this class).
  
  Reviewers do not appreciate theses that are too long or too short: If
  your thesis is much shorter than the suggested length, reviewers will
  wonder about the depth of the scientific contribution unless the
  presented results are absolutely brilliant. If your thesis is much longer
  than suggested, reviewers will get less and less inclined to continue 
  reading: part of the difficulty of presenting scientific results is
  presenting them in a limited amount of words. 
\end{answer}


\begin{question}
  How long should the different parts of my thesis (abstract,
  introduction, main text, etc.) be? 
\end{question}

\begin{answer}
  The length of the abstract must be around 150 words, independently of
  what kind of thesis you are writing. The length of the introduction
  should be around 10\% of the main text. Thus, if a bachelor's thesis
  has 30 pages, the introduction should have three to four pages (and
  not just one); a master's thesis can easily have an introduction
  with six to eight pages. The length of the conclusion should be
  around 5\% to 10\% of the main text. This leaves around 75\% to 80\%
  for the main text (because of the bibliography). You will find more
  information on the abstract, the introduction and the conclusion in 
  Section~\vref{section-abstract}.
\end{answer}




\section{Q\&A: Structuring a Thesis}

\begin{question}
  Which parts should my thesis have? Is there anything that
  \emph{has} to be there? Are there optional parts?
\end{question}

\begin{answer}
  Every thesis \emph{must} follow the same basic structure:
  \begin{enumerate}
  \item 
    Title (page),
  \item
    abstract,
  \item
    introduction,
  \item
    main chapters,
  \item
    conclusion and
  \item
    bibliography. 
  \end{enumerate}
  A thesis that misses any of these parts is not really a thesis. The
  thesis class, by the way, makes it easy to generate this structure
  since you only need to make sure that the |{document}| environment
  contains a |\chapter{Introduction}| command and a corresponding
  |\chapter{Conclusion}| command -- everything else will be inserted
  automatically at the right places and the class will give you an error
  message if anything is missing.
\end{answer}


\begin{question}
  Which main chapters should my thesis have?
\end{question}

\begin{answer}
  While the introduction itself also has a fixed structure, see
  Question~\ref{qu-intro-stru}, there is no fixed structure for the  
  main part. The number of chapters and sections really depends on
  the contents of your thesis. Usually, it is a good idea to have two
  to five main chapters (in addition to the introduction and the
  conclusion) and each can have anything between zero and ten
  sections. However, you should \emph{never} have only a single
  section in a chapter. In such a case, remove the section command.

  It is customary and useful to establish the structure of the thesis
  before you write the actual text. That is, use the |\chapter| and
  |\section| commands to create chapters and  sections without any (or
  with only little) text, but with the tentative chapter and section
  titles already in place. Then create a more or less empty thesis and
  discuss the resulting table of contents with your adviser.
\end{answer}


\begin{question}
  How important is the table of contents?
\end{question}

\begin{answer}
  The table of contents is one of the most important parts of your
  thesis.  \emph{Have a close look at it, repeatedly!}

  The ideal table of contents tells you what happens in the thesis. It
  is tough to actually get this information into a table of contents,
  but this is  what you should strive for: Organize the titles so that
  the really tell a story all by their own. As an example, consider
  the table of contents of this document. Even without having read
  anything, a reader will see immediately see that this text has two parts
  or aspects: A part that obviously serves as some sort of “user's
  guide” on the thesis class and a part that contains a lot of
  “questions and answers” on how to write a thesis. 
    
  When you read the table of contents in isolation, you will also
  easily notice inconsistencies: For instance, it often happens that
  one section starts with an article (like “The Main Experiment”)
  while another starts without one (like “Main Results”). Looking at
  the table of contents makes it easy to spot such inconsistencies
  (you should, of course, correct them). As an example, consider once
  more the table of contents of this document. It is, of course, no
  coincidence that \emph{all} sections in the second part start with
  “Q\&A”. A less obvious non-coincidence is the fact that all sections
  of the first part start with the present participle of a verb and
  that some words get repeated: I adjusted the section names at a
  relatively late stage when writing this text to ensure this
  consistency -- after I had had a look at a previous version of the 
  table of contents. 
\end{answer}



\section{Q\&A: Abstract, Introduction and Conclusion}
\label{section-abstract}

For any beginner, there is a very confusing aspect when it comes to
\emph{summarizing} the thesis: You have to do it three times! Once in
the abstract, once in the introduction and once more in the
conclusion. In the following, we have a look at why this is the case
and how these three summaries differ.

\begin{question}\label{qu-abstract1}
  How long should the abstract be? I have heard it should be half a page?
\end{question}

\begin{answer}
  The abstract should \emph{always} be exactly one paragraph of 100 to
  200 words (this is true for \emph{every} abstract, even for a thick
  book). 
\end{answer}


\begin{question}\label{qu-abstract2}
  What is the purpose of the abstract?
\end{question}

\begin{answer}
 The abstract summarizes and explains everything that happens in the
 thesis. Its job is to help someone decide ``whether the thesis contains
 something that I am currently interested in'' by just reading the
 abstract -- this abstract is often shown for instance on web pages
 without showing the whole thesis. This means that if after reading the
 abstract someone thinks ``sounds highly interesting, but I would have
 to read the thesis to know what, exactly, is shown,'' then the
 abstract is \emph{bad} since it leaves the reader wondering. If after
 reading the abstract someone thinks ``I do not need to read the thesis
 since it does not address the problem I am interest in,'' then the
 abstract is \emph{good} (provided, of course, that the thesis really
 does not address the problem\dots).
 
 This means that you must \emph{name your results in the abstract.} Of
 course, you cannot (indeed, should not) go into any details, but an
 abstract should not include just a sentence like “In the thesis, new
 approaches are tried.” Rather, write “In the thesis, three new
 approaches are tested: the foo-to-bar conversion, the bar-to-foo
 conversion and the bar-bar method. The evaluation shows that the first
 two outperform previous methods by up to 5\% on average.” 
\end{answer}


\begin{question}
  And what is the job of the introduction?
\end{question}

\begin{answer}  
  The job of the introduction (“\foreignlanguage{german}{Einleitung}”
  in German) is to introduce 
  the reader to the subject, 
  but also to \emph{explain all results obtained in the thesis.} This
  means that, after having read the introduction, the reader should
  \emph{know all major results obtained.} You may wonder what you should
  then write in the main text -- the answer is, of course, that the main
  text has the job of ``proving'' that all the results you claim in the
  introduction actually hold.
\end{answer}


\begin{question}\label{qu-intro-stru}
  Should the introduction have a special structure?
\end{question}

\begin{answer}
  It is customary to start the introduction with a motivation that explains
  what the subject is and why it is interesting. Put this right at the
  beginning, usually without a special section title like
  “Motivation”. Just start explaining. Then, however, there should
  always be three sections, starting with “Contributions of this
  Thesis” or “Results of this Thesis” (“Beiträge dieser Arbeit” or
  “Ergebnisse dieser Arbeit”), then a section on “Related Work”
  (“Verwandte Arbeiten“) and then “Structure of the Thesis” (“Aufbau
  dieser Arbeit”). Do not deviate from this structure (unless your
  adviser explicitly tells you to).
  
  The “Contributions of this Thesis“ section should explicitly state all
  of the main results -- be aware that a very busy professor may
  actually skip the main text (sadly this does
  happen) and will \emph{only} be aware of what you did from what is
  written here. Note that the contributions of a thesis are not always
  earth-shattering new fundamental breakthroughs. Indeed, they almost never
  are. Explain in clear words what you did and which scientific
  contributions you made. This may included:
  \begin{itemize}
  \item New results you obtained: This can include new mathematical
    statements, algorithms, experiment results, setups and more.
  \item Failed approaches: It is often scientific highly interesting to
    know that an approach does \emph{not} work, so that other people do not
    need to repeat the approach once more.
  \item Literature work: It can be a scientifically nontrivial
    endeavor to read and then systematically present the relevant
    literature on a certain topic.
  \end{itemize}
  
  The “Related Work” section is the place where you point out what is
  already known about the subject and what other approaches have been
  tried. Here, you explain and cite other people's works. Do not be shy
  about this, even if you feel other approaches are better -- that does in
  no way belittle \emph{your} approach.
  
  The “Structure of this Thesis” section should explain, usually in one
  paragraph, how your thesis is structured. There is no need to say
  things like ``It starts with an introduction'' or ``The thesis
  concludes with a summary and an outlook'' -- we know and expect
  this. The interesting question is how the main part is structured. Try
  to make this text interesting and do not just list the titles of the
  sections: Why does a certain chapter come first? Can different parts
  be read independently? Where are the main results to be found?
\end{answer}



\begin{question}
  Why do I have to write a conclusion? I have summarized the text
  already twice (in the abstract and in the introduction)! Summarizing
  it again makes no sense!
\end{question}

\begin{answer}
  You \emph{always} include a conclusion in the thesis. It can be short
  (5\% to 10\% of the text), but it should be present. The job of the
  conclusion is to summarize the main text once more, but the
  difference to the previous two summaries is that the conclusion is
  addressed to someone who \emph{has already read the thesis,} which
  allows you to go into much more detail in the conclusion. Here you
  can write things like ``The technically most difficult aspect was
  the proof of sub-case 4 of Lemma 3.1, where\dots'' or ``Especially
  the experiments IV and VIIa to VIIc proved difficult since\dots,''
  which would make little sense earlier. The conclusion is the place
  to describe the highlights and problems encountered in the thesis
  ``with hindsight'' while the introduction needs to summarize things
  for someone not yet familiar with the subject.

  Including an outlook is optional, if you do not include it, entitle
  the section just ``Conclusion'', otherwise “Conclusion and
  Outlook”. The outlook should really be an 
  outlook in the sense ``where do we go from here.'' Do \emph{not} use
  the outlook to list all the things that you wanted to do in your thesis,
  but did not find the time to do. In these cases, reviewers may get
  the impression that ``the student seems to have been a bit lazy,
  considering the many things she or he did not manage to do.'' It is
  then better not to include an outlook.  
\end{answer}




\section{Q\&A: Writing Scientific Text}

Writing the main part of your thesis, that is, writing the
\emph{actual text} can be daunting. Your thesis should be a “scientific
text”, but what does that mean, exactly? Surely, the text will be
complicated since science is complicated -- should you actually try to
write a “complicated text”? Or should you strive to simplify things
and explain your topic in layman's terms? In the following, we have a
look at how “scientific text” is written (in particular, in
Question~\ref{qu-we}, we have a look at why \emph{we} have a look).


\begin{question}
  I have real trouble writing anything: I cannot even write a good
  first sentence for my text! How will I ever write 30 pages?!
\end{question}

\begin{answer}
  The first sentence is often the hardest sentence to write -- so do
  not start with it! There is no need to start the text from the
  beginning with the perfect first sentence. Instead, start anywhere
  where you already know what you would like to say; the important
  lead sentences can come much later. This leads to an important
  trick: \emph{Never stop writing at the end of a 
    chapter or section, always write a few sentences for the next
    chapter or section before you call it a day.} Once you are in the
  flow, you will find it relatively easy to write text and also to
  write the starting sentences of a new chapter. When you continue
  writing on the next day or the next week, it will be much easier to
  just add something to text that is already there.
\end{answer}


\begin{question}
  I still have trouble writing anything: I need hours to write a
  single good sentence!
\end{question}

\begin{answer}
  \emph{Never} try to write “the perfect text”. Good texts do not
  arise from good writing, but from good revising: You should
  constantly go over your text and revise it, reformulate things,
  remove parts and add new ones. At the beginning, a lot of “movement”
  happens and the text changes constantly, but after some time the
  text gets better and better and the changes less and less frequent.
  As a rule of thumb, \emph{no single sentence remains unchanged from
    when it was first written to the sentence in the final version.}

  Since you should and must revise everything anyway, it is not really
  important that you get the text right at the beginning. Thus, to
  overcoming the “writer's block”, simply  write \emph{something,} no
  matter how bad it is. Once \emph{some} text is there, it is usually
  much easier to add more to it and to improve it.

  Of course, this means that you will need \emph{time} to go over the
  text repeatedly and to improve it. Do not assume that you can write
  a thesis in a week: While it might be physically possible to write
  down 30 pages in a week, it is \emph{not} possible to do five
  revisions of these 30 pages in a week. The more time you have for
  revisions, the better. 
\end{answer}


\begin{question}
  How should I start my first chapter? 
\end{question}

\begin{answer}
  Your whole thesis and each part of it can follow a simple pattern:
  Always start with a summary, followed by the details, followed by a
  wrap-up. On the top level, the title of the thesis actually
  summarizes the whole thesis! -- and it is followed by the thesis
  text. This means, in particular, that the thesis title should be
  highly informative and should really tell us what is done in the
  thesis -- do not try to be too funny or whimsical here. On the next
  level, the whole text starts with an abstract \emph{and} an
  introduction, both of which summarize the text, followed by the main
  text, followed by the conclusion.

  Now in answer to the question: You use the \emph{same} pattern also
  for each chapter. This means that in each chapter the text before
  the first section is a summary of what happens in that chapter. This
  summary should not be a section-by-section account of what comes
  next, but rather an introduction to the chapters topic with an
  overview of what we should expect in the chapter.

  You will hopefully no longer be surprised at this point to learn
  that each \emph{section} in turn should also start with a short
  introduction and even summary of the section text: Try to make sure
  that the first paragraph already contains all the punchlines of the
  section.

  You can actually even apply the pattern to each paragraph, if you
  want: The first sentence of a paragraph is the most important one
  and should lead the whole paragraph. This is also a helpful rule
  when you try to decide where a new paragraph should start: It starts
  when a new idea is started.
\end{answer}




\begin{question}
  Which words and concepts do I have to explain? Does the professor
  know all this stuff? Should anyone be able to read my text?
\end{question}

\begin{answer}
  It is always difficult to decide ``what do I need to explain, what
  does the reader already know?'' Fortunately, there is a trick that
  works most of the time: Always address the text ``to yourself, one
  year ago.'' That means, everything you already knew before you
  started working on the thesis does not 
  need to be explained once more: for a thesis in computer science,
  there is no need to explain what ``a programming language'' is and
  what ``\textsc{xml}'' means; but when writing a thesis in which you
  extend a special library that you had never heard of prior to
  writing the thesis, you \emph{do} need to explain what this library
  does. In a thesis on biology, you should not start explaining what a
  cell is or even what an amino acid is, but you should explain what
  the “parsimony assumption for the evolution of phylogenetic trees”
  is.
\end{answer}


\begin{question}
  OK, but what about the introduction and especially the motivation?
  Surely, there, I should start explaining things from the beginning?
\end{question}

\begin{answer}
  Even at the very beginning in the motivation, you should not spend
  more than one sentence on stuff “that everyone knows”. Starting with
  a very broad context is known as the “Adam and Eve beginning” since
  it starts at the, well, beginning. Instead, you should (very)
  quickly come to the interesting scientific question you address in
  the thesis.

  As an example, suppose you write a thesis on the question of how
  secure are speech recognition systems like Alexa or Siri. Then you
  should not start with something like “The first serious applications
  of artificial intelligence date back to the 1960's, but people have
  dreamed about intelligent machines for a very long time.“ This is a
  nice enough sentence, but has nothing to do with the topic of the
  thesis. A much better start would be “As more and more people use
  cloud-based speech recognition in their homes to control
  their environment, more and more people also give Internet companies
  access to potentially highly sensitive information.” Note that while
  this sentence is still fairly easy to understand and not very technical
  (and thus appropriate for an introduction) it \emph{is} about the
  subject of the thesis.
\end{answer}


\begin{question}
  How do I explain what I did? Should I be modest? Or boastful? 
\end{question}

\begin{answer}
  You may neither boast about your accomplishments nor should you be
  overly modest. It is unlikely that you reinvented all of science
  with your thesis, so do not try to create this impression. However,
  students also often are so unsure of their own achievements that the
  reader has trouble determining what the actual contribution of the
  author was!

  For an example, suppose that for your thesis you were asked to find
  the longest 
  nucleotide sequence that is common to the genome of all species in
  a large genome database (like the \textsc{ensembl}
  database). Suppose that you found out that the maximal length was 
  200,123 base pairs. You should \emph{not} write sentences
  like “This thesis demonstrates that evolution conserves huge
  quantities of genetic code over evolutionary timescales.” (way too
  boastful and not really true) nor too-modest ones like “The maximal
  length of common genomes was studied.” (What was studied, exactly? 
  And who did it? What is a “common genome”, by the way?).
  You need to write, clearly, what you did: “Since the genome of
  different species is related by evolution, all species share parts
  of their genome. In this thesis, I did a  
  quantitative study of the amount of sharing by determining the longest
  primary sequence that is present
  in the genomes of all species from the public \textsc{ensemble}
  database. The longest sequence has a length of 200,123 bases pairs
  and is the coding sequence of a cell wall protein.”\footnote{I made
    all of this up! I do \emph{not} wish to read something like “As
    pointed out by (Tantau, 2019), all species share 200,123 base
    pairs” in future papers\dots}
\end{answer}


\begin{question}
  I produced no new results! What do I do now?!
\end{question}

\begin{answer}
  Already ``writing up things nicely that were already known''
  can be an important scientific contribution. Likewise, having done an
  in-depth literature research is important for science. Naturally, if
  you have made some discoveries, explain them in the introduction and
  in the main text, but science is as much about known things as it is
  about new things.

  By the way, it is also important to point out things that did
  \emph{not} work out. Knowing that a certain approach does not work
  can be a more important scientific contribution than knowing that
  something gives a small improvement over previous methods: If you
  have shown that an approach does not work, other people will not
  waste time trying this approach again.
\end{answer}




\begin{question}
  How do I write “scientifically”? Should I use complex sentences to
  describe complex topics and simple sentences for simple topics?
\end{question}

\begin{answer}
  A good “scientific” text is a text that is easy to understand -- not
  a text that uses grand words and involved phrase
  constructions. Indeed, \emph{especially} when a subject is hard to
  understand, it is \emph{particularly} important to describe it in a
  \emph{simple} way. This does not mean that one must \emph{simplify}
  things, but just that the description should be clear and easy to
  follow -- and, spoiler alert, a simple sentence structure
  and simple words are easier to understand that a convoluted
  structure with much technical jargon.

  As an example, suppose you wish to explain the concept of public key
  cryptography. A seemingly very scientific -- but in fact just bad --
  explanation might be “Public key cryptography is based on
  private–public key pairs $(k_e,k_d)$ that encrypt and decrypt plain 
  and cover text (and \emph{vice versa}) using cryptographic hash
  functions with back-doors.” While perfectly true, this convoluted
  sentence sheds almost no light on the question of what public key
  cryptography is or does or how it works. Do not put such sentences
  into your thesis.

  What you should do instead is to use plain words in general and the
  scientific words where needed and where they are
  appropriate. Likewise, you should split up sentences when they
  become too long. Here is a much better explanation of public key
  cryptography: “Public key cryptography allows two persons who have
  never met to exchange messages over a public 
  network in the presence of an eavesdropper. It is based on
  mathematically transforming messages by first \emph{encrypting} them
  with a \emph{public} key and then later on \emph{decrypting} them
  with a \emph{private} key.”

  Here are some further some simple hints:
  \begin{itemize}
  \item Simple is better than complicated. When in doubt, use the
    simpler word or simpler structure: Instead of “the addition of the
    elements in the set~$M$ yield~$10$” write “the sum of~$M$ is
    $10$”. Instead of “Assuming that $A$ is an influencing 
    factor for how large the standard deviation of $B$ is, we may
    conclude that the correlation between $A$ and $B$ is actually
    based on a causal effect of $A$ on $B$” write “If $A$ affects how
    strongly $B$ varies, $A$ causes $B$.“

    Especially in German text, avoid too complicated sentences.
  \item Vary the sentence length. As the previous sentence shows, 
    short sentence can have a lot of “punch” since the reader can read
    and understand them at a glance, while longer sentences -- like
    the present one -- allow you to present longer thoughts. Again, if
    in doubt, try a short sentence: Instead of “The very last finding
    in the just-pre\-sen\-ted analysis rather clearly shows that
    one can no longer support the assumption that the previous
    findings are correct” write “This is wrong.”
  \item Use short, simple words, when possible. This is especially
    true in German texts. This does not mean that you should replace a
    scientific term by something seemingly simpler (it would be
    utterly wrong to replace “nondeterministic” by “random”) but that no
    complex words are introduced where simple ones would do. Instead
    of “The possibility exists to deterministically simulate the
    totality of the computational branches done by nondeterministic
    general Turing machines” write “Nondeterministic Turing machines
    can be simulated by deterministic ones.”
  \item Read a book on how to write scientific text, like then one by
    \textcite{Alley1996}.  
  \end{itemize}
\end{answer}

\begin{question}\label{qu-we}
  In scientific texts, no one uses “I”, it is always “we”. Should I
  also do this? (Should \emph{we} also do this?)
\end{question}

\begin{answer}
  The question “Should we also do this?”\ already shows
  that it is ridiculous to  always use “we” when “I” is actually
  meant. No, you should not always use “we”. Rather, the basic rules
  are as follows:
  \begin{itemize}
  \item In a single-author paper (such as your thesis), the word “we”
    always refers to “the reader and me” (that is, “you and me”). Thus,
    you write “we will see later on” since you mean “you (and I) will
    see later on” and certainly not “I will see later on”.
  \item Since “we” means “you and me”, you cannot use “we” when the
    reader does not feel addressed. You cannot write “We think that
    this is a good approach” since you force an opinion on the reader
    (who may feel that this is a terrible approach). Do not use
    phrases like “we think”, “we believe”, “we hope” and so on. You
    also cannot write “we did the following three experiments” since
    you did the experiments, but not the reader.
  \item In all of the cases where you cannot write “we”, I recommend
    writing “I” instead. “I believe that this is a good approach” is a
    clear statement with which one may or may not agree, but it is
    correct and perhaps even bold. “I did the following three
    experiments” is a simple factual statement and there is no reason
    not to write it.

    Some students (and even professors) fear that too many “I” in the
    text make the text look narcissistic and there is a certain truth
    to this. Try to use “I” mainly in the part “Contributions of this
    Thesis”, where it is your job anyway to make clear what \emph{you}
    did. In other parts, you can often rephrase things slightly:
    Instead of “I did three experiments to test whether pigs can fly”
    you can also write “for this thesis, three experiments were done
    to test whether pigs can fly“ or “pigs cannot fly according to the
    three experiments done in the context of this thesis” or “the
    three pig-flight experiments of this thesis show that pigs do not
    fly” or one of many other variations.

    However, I recommend that you do \emph{not} write “The author of
    this thesis has done three experiments” since addressing yourself
    as a third person is bad style in my opinion. Also, I think it
    would be really cool to read “I did three experiments to test
    whether pigs can fly” in a future thesis\dots\footnote{Assuming
      the ethics committee approved the experiment.}
  \end{itemize}
\end{answer}


\section{Q\&A: How to Write Mathematical Text}

Writing mathematical text is, unfortunately, as tricky as writing
normal text: There is the mathematical \emph{language,} the way
mathematicians phrase things, but there is also the problem of
typesetting mathematical text and of getting all the symbols
right. Mastering both -- the phrasing and the typesetting -- is not
easy and cannot be fully explained in a few paragraphs. But I can, as
always, give you some hints to get you started.

\begin{question}
  For mathematical text, should I use as many formulas and symbols as
  possible? Or should I perhaps try to phrase everything while using
  as little symbols as possible? Does this matter?
\end{question}

\begin{answer}
  There is no “right” amount of mathematical formalism: It really
  depends on what you describe. Beginners tend to use too much
  formalism and tend to write things like “Assuming that $A \cap B =
  \emptyset$, then\dots” instead of “If $A$ and $B$ are disjoint,
  then\dots”, but sometimes lengthy explanations are much easier to
  phrase using symbols: Instead of “Let $x$ be the limit for $n$ going
  to infinity of the series whose $n$th element is $e^{-n}n^2$” write
  “Let $x = \lim_{n\to \infty} e^{-n}n^2$”. The only general rules you
  should be aware of are:
  \begin{itemize}
  \item Every text in mathematical mode (anything surrounded by |$|'s %$
    in \TeX) should be a self-contained formula. Never write “let $n$ be
    $x+$ two divided by $4$”. Write “let $n = (x+2)/4$” or “let $n$ be
    $(x+2)/4$” or “let $n$ be $x$ plus $2$ and then divided by $4$”
    (although this is too lengthy).
  \item Every mathematical text \emph{must} be in mathematical
    mode. For instance, do not write “|Let n be an integer.|”, but
    “|Let $n$ be an integer.|”.
  \item Read the book \foreignlanguage{german}{\citetitle{Beutelspacher2009}} by
    \textcite{Beutelspacher2009}. 
  \end{itemize}
\end{answer}


\begin{question}\label{qu-math-env}
  There are so many environments in \TeX\ for mathematical text! Which
  one should I use?
\end{question}

\begin{answer}
  There are many possible environments for displaying mathematical 
  text (like |{array}|, |{align}|, |{equation}| and more) on a separate
  line. I recommend always using |{align*}| and to use |{align}|
  (the numbered version) when a numbering is really essential. Use
  other environments only when |{align}| lacks a specific feature.
  Most of the time, however, you should just inline the mathematical
  text using~|$|'s. %$
\end{answer}


\begin{question}\label{qu-math-font}
  Which font should I use for my special mathematical object?
\end{question}

\begin{answer}
  The rule is that \emph{anything variable} should be in italics,
  while everything \emph{constant and fixed} should be upright. For
  instance, in “$\sin^2 x + \cos^2 x = 1$” the text “sin” is upright
  since there is “only one sine function” and also the numbers $2$ and
  $1$ as well as the plus sign and the equality sign are all
  upright. In contrast, $x$ is in italics since this variable can
  represent an arbitrary (variable) number.

  Most of the time, \TeX\ will get these things right automatically,
  but there are cases where you may need to help:
  \begin{itemize}
  \item A simple uppercase letter is interpreted as a variable, which
    is not correct, when the letter is actually part of a name. For
    instance, if you write |$P = NP$| then \TeX\ thinks “the user
    wants me to typeset that the variable $P$ equals $N$ multiplied
    by~$P$; so I will typeset $P$ and $N$ in italics and maybe add a
    bit of space between $N$ and $P$”, resulting in “$P=NP$”. What was
    actually meant was “the complexity class $\mathrm P$ equals the complexity
    class $\mathrm{NP}$” and should have looked like this: “$\mathrm P
    = \mathrm{NP}$”. The trick is to use the |\mathrm| command, which
    tells \TeX\ to typeset something in an upright font:
    \begin{Latex}
$\mathrm{P} = \mathrm{NP}$
    \end{Latex}
  \item In addition to |\mathrm| there is also |\mathit|, which is
    needed for longer variable names in some fonts: Suppose you have a
    variable called “fit”. Now, with the Computer Modern fonts, just
    writing |$fit=1$| makes \TeX\ think “the
    users wants me to typeset $f$ times $i$ times $t$ equals $1$, so I
    better add a bit of space between all the letters”. However, this
    is just \emph{one} name, so we must actually write
    |$\mathit{fit}=1$| to get “$\mathit{fit}=1$”. 
  \end{itemize}
\end{answer}


\section{Q\&A: Language and Spelling}

\begin{question}
  Should I write my thesis in English or in German?
\end{question}

\begin{answer}
  If you are a German speaker, write your thesis in English
  \emph{only} if your English is pretty good. Students sometimes tell
  me they want to write their thesis in English “in order to
  practice”. While this seems like a good reason, it is actually a bad
  idea: Remember the introductory story from the beginning of the
  chapter? A thesis written in very bad English is a pain to read --
  and you do not wish to pain the person who grades it.

  There is no reason to write a thesis in English unless you have a
  good command of the language. If you wish to learn English, go take
  a course, take a vacation in London or write a shorter paper
  together with someone who \emph{is} good at English.

  (You should, of course, always try to improve your English whenever 
  possible since it is the language of science. But you should
  \emph{not} try to turn your adviser into a language coach.)
\end{answer}


\begin{question}
  How do I avoid spelling mistakes?
\end{question}

\begin{answer}
  Well, use a spell checker. Quite frankly, when I get a thesis with
  several spelling mistakes that any spell checker would have caught,
  my pulse actually rises as I think “why the $\langle$beep$\rangle$
  do I have to read this $\langle$beep$\rangle$ where the author could
  not be $\langle$beep$\rangle$ bothered to run a spell checker just
  once?” You will \emph{really} annoy your reviewers with obvious
  mistakes.

  However, in addition to the mistakes that any spell checker will
  catch, there are of course many grammatical mistakes that one can
  make. In English texts, the problem is typically the word order that 
  Germans get wrong; in German texts, students like to add and remove
  commas at random places throughout the text. Well, there a rules for
  all of this. If you could not be bothered in school to learn them,
  learn them now -- there is no time like the present. You \emph{must}
  follow them.

  Of course, most of the time students do not make grammatical
  mistakes because they simply do not know where the comma should go,
  but because they just make a mistake and do not catch it later
  on. The trick is to \emph{read, re-read and re-re-read} the text as
  often as necessary until almost all mistakes are gone.

  It helps to read the text slowly.

  It helps to have someone else read the text.

  It helps to read the text backwards (yes, really: you cannot read
  quickly in this case).

  Whatever you do, a well-written text should not have more than one
  spelling mistake every few \emph{pages} -- and not one every few
  \emph{lines} let alone one every few \emph{words}.
\end{answer}


\begin{question}\label{qu-titles}
  In English titles most words seem to start with an uppercase
  letter. Is this normal?
\end{question}

\begin{answer}
  In English, it is customary to spell (almost) all words with
  uppercase letters in titles including chapter and section
  titles. So, the title of the first section of this text is
  ``Contributions of this Thesis'' rather than ``Contributions of this
  thesis''. However, as can be seen, some words like “of” or
  ``this'' are not capitalized: Usually, conjunctions and prepositions
  are not capitalized. To a certain degree, you can choose, how many
  words you capitalize: some people capitalize everything with the sole
  exceptions of ``a'', ``an'' and ``the'', other people do not to
  capitalize at all, but this is unusual. Whatever ``amount of
  capitalization'' you choose, stick with it! It is plain \emph{wrong}
  to write ``Contributions of this Thesis'' and then ``Structure of This
  Thesis''. 
\end{answer}


\begin{question}
  Do I have to also capitalize words in titles in German?
\end{question}

\begin{answer}
  In German you have to capitalize according to the German spelling
  rules. So, you would write ``\foreignlanguage{german}{Untersuchungen
    zum genetischen Code}'' and not
  ``\foreignlanguage{german}{Untersuchungen zum Genetischen Code}'',
  which would be plain wrong -- even in the title of the thesis. 
\end{answer}



\begin{question}\label{qu-quotation-marks}
  Which quotation marks should I use? And how do I enter them?
\end{question}

\begin{answer}
  Getting quotation marks right used to be somewhat complex in \LaTeX;
  these days the main problem is to know which one are the right
  ones:

  \tikzset{ex/.style={scale=2,inner sep=0pt}}
  \medskip
  
  \begin{tabular}{llcc}
    \uzlhline
    \uzlemph{Language} & \uzlemph{Example} &
    \uzlemph{“Opening Form”} & \uzlemph{“Closing Form”} \\
    \uzlhline
    American English & “Example” & 66 & 99 \\ 
    British English & ‘Example’ & 6 & 9\\ 
    German first alternative & „Beispiel“ & 99 & 66 \\ 
    German second alternative & »Beispiel« & »  & « \\
    \uzlhline
  \end{tabular}

  Anything else is just wrong.

  I strongly recommend that you find out how you can directly enter
  the Unicode characters for the quotation marks of your choice and
  directly enter them in the text.
\end{answer}


\section{Q\&A: Creating Figures and Tables}

\begin{question}
  How many figures should my thesis have?
\end{question}

\begin{answer}
  I once knew a person who complained to me that in her PhD thesis on
  analytic \emph{geometry} she did not have a single graphic. The
  reason was that all of her thesis was about infinite dimensional
  spaces and these are a bit hard to visualize in two dimensions.

  A thesis can have anything between zero and dozens of
  figures. However, make sure that “each figure counts”: Readers pay
  special attention to figures and if they do not understand them or
  cannot read them clearly, they will be much more annoyed than by a
  poorly phrased paragraph. It also takes rather a lot of time and
  energy to create a good figure, so think twice before adding a
  figure. Finally, too many figures tend to create the impression that
  one is not reading a thesis, but a manual (or, worse, is looking at
  a picture book): If you have a lot of 
  figures, it may be better to leave some of them out and to
  concentrate on the really important ones that best convey the main
  message. 
\end{answer}

\begin{question}
  What program should I use for creating graphics?
\end{question}

\begin{answer}
  I recommend not to use a separate program, but to use Ti\emph kZ and to
  create the graphics right inside the \LaTeX\ manuscript, see
  Section~\vref{section-tikz}. But this 
  recommendation may be slightly biased (I programmed Ti\emph kZ\dots). 
\end{answer}

\begin{question}\label{qu-scaling}
  Thanks for Ti\emph kZ, but I still wish to include graphics that I
  produce with my favorite drawing program. Surely, this is no problem?
\end{question}

\begin{answer}
  Technically, it is no problem to include \textsc{pdf}, \textsc{jpg}
  or \textsc{png} files, see Section~\vref{section-external}. However,
  there is one \emph{big danger: Scaling!} Scaling can be evil for two
  reasons:
  \begin{itemize}
  \item
    Scaling changes not only the size of a graphic, but also the line
    widths. If you scale down a vector graphic by a factor of ten to
    fit it into your thesis, all lines will be so thin that they are
    almost invisible. The other way round, if you magnify a graphic
    using scaling, the lines may become so thick that they look almost
    comical.
  \item
    Scaling does even worse things with font sizes. The correct font
    size for use in figures in a thesis is 10pt. Good luck getting
    this font size correct when including an external graphic -- text
    in graphics will almost always be way too small or way too
    large. Unintelligibly small text \emph{really} annoys reviewers
    and we see it all the time in tables. Too big text makes the
    thesis look like a child's book. You do not want your thesis to
    look like a child's book.
  \end{itemize}
  Apart from making it hard to get the font sizes right, getting the
  fonts \emph{themselves} right is even harder. Good luck creating a
  graphic with the correct mathematical symbols from the Alegrya font
  used in it.

  Ideally, only include external pictures and create graphics with
  text in them  using Ti\emph kZ. If this is not an option, try to
  setup things so that no scaling is done and that the font size is
  10pt and the normal line width is 0.4pt to 0.6pt.
\end{answer}


\begin{question}\label{qu-captions}
  How long should captions (for figures, tables, graphics, etc.)\ be?
\end{question}

\begin{answer}
  I recommend using \emph{very long captions} (up to five lines) with
  figures and tables. The reason is that people look at figures and
  tables before they look at the main text. Figures
  should be ``understandable all by themselves'' and the caption has the
  job of ensuring this. In other words, make figures ``as independent
  of the main text as possible.'' Naturally, you should still
  reference all figures in the text; \emph{never} add a figure that is
  not mentioned in the text. 

  The caption of Figure~\vref{fig-ode1} demonstrates the problem
  with a short caption: Most readers will have no idea what this is
  about. A much better caption is the one of Figure~\vref{fig-ode2}.
\end{answer}

\begin{figure}[htbp]

  \centerline{\textcolor{black!20}{\vrule width3cm height 3cm}}

  \caption{The depicted square is intended as an ``ode'' (that is, as
    a praising and -- in this case -- slightly ironic reference) to
    the famous painting \emph{Black 
      Square} by Kasimir Malewitsch from 1915, which depicts a black
    square. The above square is clearly not black, but just gray, and
    is hence intended as a whimsical reference.}
  \label{fig-ode2}

\end{figure}



\section{Q\&A: How to Cite and What to Cite}


Several well-known politicians recently lost their jobs because of
some missing or incorrect citations in their PhD theses. Clearly,
getting these citations right must be something important. Well, there
is a very simple basic rule that the politicians violated: For every
word of the text, it must be \emph{crystal-clear who had the idea for that
  word.} There is nothing wrong, at all, with citing even a whole page
from another paper -- as long as it is clear that this is text from
another paper! 

\begin{question}
  When and where do I have to add citations?
\end{question}

\begin{answer}
  In accordance with the basic rule (“for each word of the thesis it
  must be crystal-clear who had the idea for that word“) whenever an
  idea or a concept is from someone else, a citation is added. You add
  these citations next to the point of the idea or the text using the
  |\cite| command or related commands such as for instance
  |\citeauthor|, see the \cite{biblatex} manual for more details.
\end{answer}

\begin{question}
  Where can I get bibliography records (the Bib\TeX\ entries)?
\end{question}

\begin{answer}
  Many servers allow you to download Bib\TeX\ records for books and
  articles, but \emph{be warned that these very often have very poor
    quality.} You \emph{must} go over them by hand and often you will
  have to correct them. Ask your adviser if you are unsure. 
\end{answer}


\begin{question}
  How many entries should there be in the bibliography?
\end{question}

\begin{answer}
  There are no lower or upper bounds, but reviewers use the entries in
  the bibliography\footnote{Depending on the design used to typeset
    this text, the word \emph{bibliography} gets hyphenated here in a
    very strange way: bib-li-og-ra-phy. It turns out, this is the
    correct English hyphenation, which I find weird since the word is
    clear composed of \emph{biblio} (book) and \emph{graphy}
    (writing). Luckily, \TeX's 
    hyphenation rules are pretty good and get this right.} as a
  measure of how much literature research has 
  been done. If there are only one or two (or even zero) scientific
  articles, one old textbook and a lot of web pages that are cited, I
  would seriously doubt that a lot (or, for that 
  matter, \emph{any}) literature research has been done. On the
  other hand, if there is an endless list of entries in the
  bibliography, only very seldom did the student actually read all of
  these papers -- more likely, most entries in the list have just been
  copied from somewhere else without the student's actually having
  looked into them. 

  The bottom line is that you should do a solid literature research
  (you should spend \emph{at least} two \emph{weeks} on
  that!) and then cite the five to fifteen papers and books that are
  really relevant and that you have actually read.
\end{answer}


\begin{question}
  Do I have to read all papers that I cite?
\end{question}

\begin{answer}
  Yes, you do -- perhaps not completely, but you must read the part of
  the text to which you refer in your citation. Resist the temptation
  to just copy citations from other papers.
\end{answer}


\begin{question}
  Can I cite Wikipedia? 
\end{question}

\begin{answer}
  Yes, you can, but only if there is not a better scientific source
  available. For example, if you wish to cite the notion of a
  “Turing machine”, you should not point the reader to the
  corresponding article in Wikipedia, but to an article in a standard
  textbook on theoretical computer science. 
\end{answer}



\chapter{Conclusion}
% In a German thesis write: \subsection{Zusammenfassung und Ausblick}


% !!!!!!!!!!!!!!!!!!!!!!!!!!!!!!!!!!
% !!! Your action is needed here !!!
% !!!!!!!!!!!!!!!!!!!!!!!!!!!!!!!!!!
%
% Replace the following with your conclusion



This template document got much longer than I had initially intended
with more and more hints and comments becoming part of the text. The
reason is, of course, that writing a thesis is not easy since there
are a \emph{lot} of things to consider. However, you have six months
to write your thesis, so you stand a decent chance to get most things 
right.

Do some great scientific research now and report on it in a thesis
that is a pleasure to read. 




% Normally, the bibliography comes next at this point. Do *not* (try
% to) include further indices and tables like an index or
% a list of figures or a list of tables or such things. Nobody
% actually uses them and they just use up space. 
%
% You *can* however include a glossary, if this seems appropriate. It
% goes here as an unnumbered chapter. Most thesis will *not* need a
% glossary: a well-written text (re)explains strange words and
% concepts as necessary. However, there are situations where a
% glossary may be helpful.














%%%
% 
% Bibliographies
%
%%%
%
% The uzl-thesis class will load biblatex for the bibliography
% management. This is a powerful package, see its documentation for
% details. The styles will be setup correctly and automatically by
% choosing one of the two style keys as described earlier.
%
% In order for the bibliography to work, run latex in the following
% order (which is the standard order):
% 
% > lualatex thesis-example
% > bibtex thesis-example
% > lualatex thesis-example
% 
% Add BibTeX files using \addbibresource or use the {bibtex entries}
% environment (see below).
%
%%%
%
% Although everyting is normally setup automatically, you can change
% the options passed to biblatex using the key 'biblatex';
% for instance,
%
%   \UzLThesisSetup{biblatex={firstinits=false}}
%
% will switch off shortened first names. Normally, you will not need
% this key in your preamble. 
% 
% Note that the bibtex program is used as the 'backend' of biblatex
% by default (rather than biber, which is the preferred program of
% biblatex). This means that you can (and must) run *bibtex* after you
% have run lualatex on your thesis. If you wish to use biber instead
% of bibtex, say 'biblatex={backend=biber}'. 
% 
%%%
%
% The following environment is optional. It allows you to keep the
% bibtex entries for your thesis right here in the thesis file. What
% happens is that each time this tex file is processed, the contents
% of the following environment gets written to the file
% \jobname-bibtex-entries.bib (this file gets overwritten each
% time). Independently, \addbibresource{\jobname-bibtex-entries.bib}
% is always called if the file \jobname-bibtex-entries.bib
% exists. 
%
% In result, you can edit and keep the bibliography's bibtex entries
% right here. If you change something here, run latex, then bibtex,
% then latex once more.
%
% If you would like to manage the bibtex entries in a separate file,
% remove the below environment, delete the \jobname-bibtex-entries.bib
% file and instead write
%
% \addbibresource{filename-of-your-bibtex-file.bib}
%
% in the preamble.
%
%%%


% !!!!!!!!!!!!!!!!!!!!!!!!!!!!!!!!!!
% !!! Your action is needed here !!!
% !!!!!!!!!!!!!!!!!!!!!!!!!!!!!!!!!!
%
% Replace following example entries with the ones of your thesis.

\begin{bibtex-entries}

@Book{Knuth1986,
  author =       {Donald Erwin Knuth},
  title =        {The \TeX book},
  publisher =    {Addison-Wesley},
  year =         {1986},
}

@Book{Lamport1994,
  author =       {Leslie Lamport},
  title =        {\LaTeX: A Document Preparation System},
  publisher =    {Addison-Wesley},
  edition =      {Second edition},
  year =         {1994},
}

@TechReport{Kernighan1974,
  author =       {Brian Kernighan},
  title =        {Programming in C – A Tutorial},
  institution =  {Bell Laboratories},
  year =         {1974}
}

@Manual{Tantau2019,
  author =       {Till Tantau},
  title =        {The Ti\emph kZ and PGF Packages: Manual for version 3.1.3},
  institution =  {Institut für Theoretische Informatik, Universität zu Lübeck},
  year =         {2019},
  url =          {https://github.com/pgf-tikz/pgf}
}

@Book{Alley1996,
  author =       {Michael Alley},
  title =        {The Craft of Scientific Writing},
  publisher =    {Springer},
  year =         {1996},
  edition =      {Third Edition},
}

@Book{DowneyF13,
  author =       {R. G. Downey and M. R. Fellows},
  title =        {Fundamentals of Parameterized Complexity},
  series =       {Texts in Computer Science},
  publisher =    {Springer},
  year =         2013,
  doi =          {10.1007/978-1-4471-5559-1},
}

@Manual{biblatex,
  title =        {The \textsc{BibLaTeX} package},
  subtitle =     {Sophisticated Bibliographies in \LaTeX},
  author =       {Kime, Philip and Lehman, Philipp},
  url =          {https://github.com/plk/biblatex},
  urldate =      {2019-06-11},
  date =         {2018-10-30},
  version =      {3.12}
}

@Manual{varioref,
  title =        {The \textsc{varioref} package},
  subtitle =     {Intelligent page references},
  author =       {Mittelbach, Frank},
  url =          {http://www.ctan.org/pkg/varioref},
  urldate =      {2019-06-11},
  date =         {2016-02-16},
  version =      {1.5c}
}

@Manual{hyperref,
  title =        {The \textsc{hyperref} package},
  subtitle =     {Extensive support for hypertext in \LaTeX},
  author =       {Rahtz, Sebastian and Oberdiek, Heiko},
  url =          {https://github.com/ho-tex/hyperref},
  urldate =      {2019-06-11},
  date =         {2018-11-30},
  version =      {6.88e}
}

@Manual{babel,
  title =        {The \textsc{babel} package},
  subtitle =     {Multilingual support for Plain \TeX\ or \LaTeX},
  author =       {Braams, Johannes L. and Bezos López, Javier},
  url =          {http://www.ctan.org/pkg/babel},
  urldate =      {2019-06-11},
  date =         {2019-06-03},
  version =      {3.32}
}

@Manual{fontspec,
  title =        {The \textsc{fontspec} package},
  subtitle =     {Advanced font selection in Xe\LaTeX\ and Lua\LaTeX},
  author =       {Robertson, Will},
  url =          {http://www.ctan.org/pkg/fontspec},
  urldate =      {2019-06-11},
  version =      {2.7c}
}

@Manual{url,
  title =        {The \textsc{url} package},
  subtitle =     {Verbatim with \textsc{url}-sensitive line breaks},
  author =       {Arseneau, Donald},
  url =          {http://www.ctan.org/pkg/url},
  urldate =      {2019-06-11},
  date =         {2013-09-16},
  version =      {3.4}
}

@Manual{amsmath,
  title =        {The \textsc{amsmath} package},
  subtitle =     {\AmS\ mathematical facilities for \LaTeX},
  author =       {{The \LaTeX\ Team}},
  url =          {http://www.ams.org/tex/amslatex.html},
  urldate =      {2019-06-11}, 
  date =         {2017-09-02},
  version =      {2.17a}
}

@Book{Beutelspacher2009,
  title =        {„Das ist o.\,B.\,d.\,A.\ trivial!“: Tipps und Tricks zur
                  Formulierung mathematischer Gedanken (Mathematik für
                  Studienanfänger)},
  author =       {Albrecht Beutelspacher},
  year =         {2009},
  edition =      {Ninth, updated edition},
  publisher =    {Vieweg+Teubner Verlag},
  doi =          {10.1007/978-3-8348-9075-7},
}

\end{bibtex-entries}



% If you need to have an appendix (I advise against it), insert it
% here using, first, \appendix and then \chapter and then,
% possibly, \section. 
%
% \appendix
%
% \chapter{Technical Appendix}
%
% \section{Experimental Parameters} % possibly
%
% Again, I advise against using an appendix.


\end{document}

%  LocalWords:  LaTeX tex moretexcs Lübeck pdf uzl lualatex bibtex th
%  LocalWords:  TechReport Kernighan Lamport's Tantau's Tantau cls kZ
%  LocalWords:  Mustermann emacs oldschool pdflatex texmf utf biber
%  LocalWords:  biblatex Alphabetische Bibliographie Numerische VIIa
%  LocalWords:  varioref german Einleitung Beiträge dieser Arbeit xml
%  LocalWords:  Ergebnisse Verwandte Arbeiten Aufbau nucleotide VIIc
%  LocalWords:  ensembl amino phylogenetic Alexa Siri decrypt versa
%  LocalWords:  cryptographic pre nondeterministic deterministically
%  LocalWords:  Beutelspacher Untersuchungen zum genetischen sep llcc
%  LocalWords:  Beispiel tikz jpg png Alegrya Kasimir Malewitsch PGF
%  LocalWords:  Lamport Institut für Theoretische Informatik zu url
%  LocalWords:  Universität Springer DowneyF Downey Parameterized doi
%  LocalWords:  BibLaTeX Kime Philipp urldate Mittelbach hyperref Lua
%  LocalWords:  Rahtz Oberdiek Heiko Braams Bezos López fontspec Das
%  LocalWords:  Arseneau amsmath ist Tipps und zur Formulierung
%  LocalWords:  mathematischer Gedanken Mathematik Studienanfänger
%  LocalWords:  Albrecht Vieweg Teubner Verlag
